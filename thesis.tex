% ========================
% This is the main LaTeX
% source file for 'thesis.pdf'
% ========================

\documentclass[12pt,headA,chapB]{fiskthesis}
\usepackage{rotating}
\usepackage{pdfpages}
\usepackage{longtable,lscape}
\usepackage{aas_macros}
\usepackage{rotfloat}
\usepackage[page]{appendix}
\usepackage{hyperref}
\usepackage{ulem}

% Added by Nicole
\usepackage{lipsum}
\usepackage{titlesec}
\usepackage{sectsty}
\usepackage{tocloft}
\usepackage{natbib}
%\bibliographystyle{plain}
\bibliographystyle{apj}


% Centered title for ToC, LoF, LoT
\renewcommand{\contentsname}{\hfill\bfseries\normalsize TABLE OF CONTENTS\hfill}   
\renewcommand{\cftaftertoctitle}{} 
\renewcommand{\listfigurename}{\hfill\bfseries\normalsize LIST OF FIGURES\hfill}   
\renewcommand{\cftafterloftitle}{} 

\setlength{\cftbeforetoctitleskip}{-2em}
\setlength{\cftbeforeloftitleskip}{-2em}


\titleformat{\chapter}
  {\normalsize\bfseries} % format
  {\thechapter. }                % label
  {0pt}             % sep
  {\normalsize}           % before-code
\titlespacing*{\chapter}{0pt}{-25pt}{12pt}

\sectionfont{\normalsize}
\setlength{\footskip}{20pt}

\pagestyle{fancy}
 \renewcommand{\headrulewidth}{0.5pt}
 \renewcommand{\chaptermark}[1]{\markboth{}{}}
 \lhead{}%\protect\textnormal{\leftmark}}
 \chead{}
 \rhead{\small\thepage}
 \lfoot{}
 \cfoot{}
 \rfoot{}
 \renewcommand{\headrulewidth}{0pt} % Remove line under page number

\begin{document}

\frontmatter
% *** TITLE PAGE ***
\begin{ThesisTitle}
\Title{Cosmological Hydrodynamic Simulations of Preferential Accretion \protect \\ in the SMBH of Milky Way Size Galaxies}
\Department{Astronomy}
\Author{Natalie Nicole Sanchez}
\GradDate{June}{2016}
\end{ThesisTitle}
% *** END TITLE PAGE **


% *** APPROVAL PAGE ***
\begin{Approval}
\ATitle{Cosmological Hydrodynamic Simulations of Preferential Accretion \protect \\ in the SMBH of Milky Way Size Galaxies}
\AAuthor{Natalie Nicole Sanchez}
\Degree{B.S.}
\University{California Polytechnic State University}
\GradYear{2014}
\City{Pomona}
\State{California}
\ADepartment{Physics}
\end{Approval}

\begin{Committee}
\Name{Dr. Kelly Holley-Bockelmann}
\Role{Thesis Supervisor}
% ---------------
\Name{Dr. Andreas Berlind}
\Role{Committee Member}
% ---------------
\Name{Dr. Steve Morgan}
\Role{Committee Member}
% ---------------
\Chair{Dr. Keivan Stassun}
% ---------------
\end{Committee}
% *** END OF APPROVAL PAGE ***


\newpage
\doublespacing
% *** ACKNOWLEDGEMENTS ***
\addcontentsline{toc}{section}{Acknowledgements}
%\section*{Unnumbered Section}
\thispagestyle{empty}
\vspace{7ex}%
\begin{center}%
% \textbf{Acknowledgements}\\
\textit{ To my mother\textemdash \\
who predicted this chapter in my life \\
before I'd even started the book. }
%who made it clear I would be going to college \\
%before I even knew what it was. }
\vspace{4ex}
\end{center}
% *** END OF ACKNOWLEDGEMENTS ***


% *** TABLE OF CONTENTS *** 
\newpage
\tableofcontents
\thispagestyle{fancy}
% *** END OF TABLE OF CONTENTS ***


% *** LIST OF TABLES ***
%\clearpage
%\listoftables
% *** END OF LOT ***


% *** LIST OF FIGURES ***
\newpage
\addcontentsline{toc}{section}{List of Figures}
\listoffigures
\thispagestyle{fancy}
% *** END OF LOF ***


% *** ABSTRACT ***
\newpage
\addcontentsline{toc}{section}{Abstract}
\chapter*{\normalsize ABSTRACT}
\thispagestyle{empty}
\rhead{\small\thepage}
Using cosmological hydrodynamic simulations of Milky Way-type (MW-type) Galaxies, we explore how varying assembly histories affect the mass budget of the SMBH. We examine two MW-mass halos with differing merger histories. One is characterized by several major mergers and the other with a quiescent history. We attempt to identify the importance of merger history on black hole accretion using varying resolution and SPH codes. This study is an extension of Bellovary et. al. 2013, which analyzed the accretion of high mass, high redshift galaxies and their central black holes, and found that the gas content of the central black hole is proportional to that accreted by the host galaxy halo. In this study, we find that a merger-rich galaxy will have a central SMBH preferentially fed by the merger gas fueling the system. A quiescent galaxy, however, will have a SMBH that mirrors its host's composition. Through an investigation of the angular momentum of the gas entering these hosts and their SMBHs, we determine that merger gas enters the galaxy with lower angular momentum compared to smooth accretion, partially accounting for the preferential fueling witnessed in the SMBH of the galaxy with an active merger history. In addition, the disruptive power of mergers, particularly major merger (q \textgreater 0.3), also helps funnel the low angular momentum gas more readily to the center of the galaxy. Our results imply that galaxy mergers play an important role in the feeding of the SMBHs in the class of Milky Way-type galaxies.
\vspace{4ex}
%\end{center}
% *** END OF ABSTRACT ***


\mainmatter
% *** INTRODUCTION ***
\chapter{\normalsize INTRODUCTION}
\section{\normalsize Background}
%\pagenumbering{gobble}
\thispagestyle{empty}

Supermassive black holes (SMBHs) are thought to exist in almost all massive galaxies \citep{Kormendy2013}. In the canonical picture of BH growth, these black holes may become active galactic nuclei (AGN) during periods of high accretion and wane in periods of quiescence \citep{Alexander2005,Papovich2006,Volonteri2012}. The host galaxy's size, star formation rate, and other environmental effects may help to influence the growth of the black hole residing at its center; however, there are still uncertainties concerning the relationship between these SMBHs and their much larger host galaxies, as well as how they grow and evolve together \citep{Haehnelt2000,DiMatteo2005,Hopkins2006,Fu2008,Sijacki2009,Silverman2009,Mullaney2012}.

The M$-\sigma$ relation, which relates the SMBH's mass and the velocity dispersion of the host galaxy's central stellar population, gives some insight into the complex interplay between these objects \citep{Ferrarese2000}. A prominent trend appears, as SMBHs tend to scale with the velocity dispersion of the host galaxy bulge. The tightness of the relation is significant and can be seen over several orders of magnitudes in velocity dispersion and black hole mass \citep{Merritt2001,Graham2011,Mcconnell2013,Kormendy2013}. Scatter exists among the low mass galaxies and a deviation may appear at the high mass end, where overmassive BHs may reside \citep{VanDenBosch2007,Moster2010,Natarajan2011}. However, scatter in less massive galaxies may imply that there are several channels of black hole growth at play in the low mass end of the relation \citep{Micic2007,Volonteri2009,Reines2013,Graham2014}. One standard explanation for the M$-\sigma$ relation lies in galaxy mergers, which build up galaxies, feed SMBHs, and assemble bulges \citep{DiMatteo2005,Shen2008}. Major mergers are thought to supply gas to the central SMBH resulting in feedback which quenches star formation and affects the structure of the galaxy. 

\begin{figure}
\centerline{\resizebox{0.75\hsize}{!}{\includegraphics[angle=0]{fancymsigma}}}
\caption[M-$\sigma$ Relation]{The M-$\sigma$ relatpion which depicts the strong correlation between the mass of a central SMBH and the velocity dispersion of its host's stellar population \citep{Xiao2011}.}
\label{msigma} 
\end{figure}

% FOR SHOCKS: The simulations I am studying are run with a new implementation of Gasoline, which models hydrodynamic processes like shocks very well by using a geometric density average in the SPH force expression

The black hole mass-bulge luminosity relation was implied by the work of \cite{Dressler1988} and first illustrated by \cite{Kormendy1993}. A wealth of evidence continues to relate these two characteristics, and follow up examinations using HST observations have determined the current, best estimates yield a M$_{\rm BH}$/M$_{\rm bulge}$ fraction between about 0.0013 and 0.0023 \citep{Merritt2001a,McLure2001,Marconi2003}. 

\begin{figure}[h]
\centerline{\resizebox{0.75\hsize}{!}{\includegraphics[angle=0]{h258_faceon_crop}}}
\caption[GASOLINE h258]{Face-on and edge-on Sunriwse images of our GASOLINE galaxy, h258, which has an active, merger-rich history. At z $\sim$ 0, the halo has a mass of 7.87 $\times$ 10$^{11}$ M$_{\odot}$, the SMBH has a mass of 6.94 $\times$ 10$^6$ M$_{\odot}$, and a virial radius of 238.49 kpc. }
\label{h258face} 
\end{figure}

Major mergers between massive galaxies are thought to be efficient fueling mechanisms for bright AGN. The large influx of material due to tidal torques from the merger causes bursts of star formation and helps funnel gas directly into the center where the SMBH resides \citep[e.g.][]{Richards2006,Reddy2008,Hopkins2010}. Additionally, the most massive, highest-luminosity AGN (i.e. quasars) reside in incredibly luminous infrared galaxies where star formation is abundant, signifying that major mergers may have recently occurred \citep{Treister2012}. Distorted morphologies are often characteristics of quasar hosts, and companions can also be present around quasars, both of which are evidence that strengthen the possibility of a recent merger having affected their lifetimes. 

In many less massive and less luminous AGN, however, there is a clear lack of distorted morphologies, close neighbors, and/or other obvious merger evidence \citep{Ryan2007,Schawinski2011,Ellison2013,Hicks2013}. It is also important to note that many of these AGN exist in spiral galaxies, which are unlikely to have been recently disturbed by major mergers \citep{Springel2005,Kocevski2011}. Nevertheless, some evidence suggests \citep{vanGorkom1997,Governato2009} that disturbed galaxies may reform a disk quickly, even after a major merger as long as it is gas-rich. More recently, \cite{Treister2012} has suggested that only the highest luminosity AGN require fueling via major mergers; $\sim$90\% of AGN across all redshifts are fueled by various other mechanisms which may include minor mergers, flybys, and smooth accretion, whereby gas is directly accreted via large filaments from the ambient intergalactic medium \citep{Cox2006,Bellovary2013,Sinha2012}. 

Smooth accretion, in particular, may play an important role in fueling these low mass galaxies. Halos less than 10$^{11}$ $M_{\odot}$ can accrete filaments of unshocked gas; thereafter, gas will shock heat to the virial temperature of the halo \citep{Keres2005}. Even for massive halos, unshocked gas may still penetrate shocked regions to fuel the galaxy \citep{Brooks2007,Dekel2009,Nelson2013}. In addition, SMBH feedback, the depositing of energy and angular momentum back into the gas resevoir during accretion, also affects the overarching structure of the host galaxy \citep{Governato2009a}. Secular processes, including bar formation and disk instabilities, may also be prominent forms of accretion for these SMBHs \citep{Kormendy2013}. 

\section{Objectives}
It is clear that galaxy hosts grow through variety of channels that depend on mass, environment, and interaction history. Therefore, we want to understand how these different galaxy evolutionary paths translate into SMBH fueling mechanisms, and see how they affect the fueling gas flowing into the SMBH itself. \cite{Bellovary2013} compared simulations of three high mass, high redshift galaxies and found that while mergers and smooth accretion both efficiently build up galaxies, no particular method was more adept at feeding the SMBH. Using a similar method as \cite{Bellovary2013}, this work compares the SMBH and galaxy fueling mechanisms between two Milky Way (MW) mass galaxies. MW-type galaxies host SMBHs on the order of 10$^6$ $M_{\odot}$, which are likely the most common type of massive black hole, yet little is known about them or how they may grow \citep{Kormendy2013}. Through this examination, we hope to better understand the coevolution of SMBHs and their hosts in this class of galaxy. 

\begin{figure}[h]
\centerline{\resizebox{0.75\hsize}{!}{\includegraphics[angle=0]{h277_faceon}}}
\caption[GASOLINE h277]{Face-on and edge-on Sunrise images of our GASOLINE galaxy, h277, which has a quiescent, merger-quiet history. At z $\sim$ 0, the halo has a mass of 6.48 $\times$ 10$^{11}$ M$_{\odot}$, the SMBH has a mass of 2.40 $\times$ 10$^7$ M$_{\odot}$, and a virial radius of 223.55 kpc. }
\label{h277face} 
\end{figure}

We analyze two galaxy simulations that are similar at z = 0 but have very different merger histories. Our ``active'' galaxy, h258 (Figure \ref{h258face}), has a history characterized by major mergers, while our ``quiet'' galaxy, h277 (Figure \ref{h277face}), has a quiescent history with only minor mergers. Since these galaxies are similar to the MW in virial mass, stellar mass, and circular velocity, without a deeper examination, we may not recognize the varying histories that distinguish them. We will compare the origins of gas entering the SMBH and halo to look for clues about SMBH fueling within these galaxies. By examining the assembly of these galaxies, and their SMBH fueling, we compare the accretion rates between h258 and h277 to determine how SMBHs in galaxies like our own.


% *** END OF INTRODUCTION ***


% *** SIMULATION PARAMETERS ***
\chapter{\normalsize SIMULATION PARAMETERS}
\thispagestyle{empty}

The cosmological simulations have been run using two smoothed particle hydrodynamics (SPH) N-body tree code: Gasoline \citep{Wadsley2004} and, more recently, Charm N-body GrAvity solver, ChaNGa. 

An initial DM-only, uniform resolution 50 comoving Mpc box determined which halos would be selected for zoom-in examination, including the two halos examined in this paper. The DM-only simulation assumed a WMAP Year 3 cosmology \citep{Spergel2007} with the following specifications: $\Omega _m$ = 0.24, $\Omega _{\Lambda}$ = 0.76, $H_0$ = 73 km/s, and $\sigma _8$ = 0.77. The halos h258 and h277  were chosen for their Milky Way-mass, between 6--8 $\times$ 10$^{11}$  M$_{\odot}$, at z=0 and their active and quiescent merger histories, respectively. The halos have virial masses defined relative to a critical density, $\rho _c$, where $\rho / \rho _c$ = 100 where h258 and h277 have virial masses of $M_{\rm vir} = 7 \times 10^{11} M_{\odot}$ and $M_{\rm vir} = 8 \times 10^{11} M_{\odot}$, respectively. A recent major merger characterizes the h258 halo at z=1, while h277 has its last significant merger near z $\sim$ 3. A second ``zoom-in'' high resolution simulation was run for both of these galaxies including gas and star particles using the volume renormalization of \cite{Katz1993}, resimulating only a few virial radii from the main halo at the highest resolution. The Gasoline and ChaNGA simulations were run from z=99 to z=0 and z=150 to z=0, respectively.  
%[Make sure this is true for low res. True for high res.]

Gas can reach a minimum temperature of $\sim$100 K, in the absense of cooling via molecular hydrogen or metals. The simulation includes stochastically-modeled star formation, and once the density threshold and temperature satify conditions for star formation (2.5 amu cm$^{-3}$; T 
\textless 10$^4$ K), gas particles are eligible to form stars with an efficiency of c$^*$ = 0.1. Star particles form along the Kroupa initial mass function \citep{Kroupa2001}. Supernova feedback releases $10^{51}$ ergs of thermal energy and affects a ``blastwave'' radius determined by the equations of \cite{Ostriker1988} In the affected region, cooling turns off for a time relative to the expansion phase of the SN remnant also determined by the blastwave equation. SN Ia and II from \cite{Thielemann1986} and \cite{Woosley1986} are adopted, respectively, and implemented through the \cite{Raiteri1996} method, which uses the stellar lifetime calculations of the Padova group \citep{Alongi1993, Bressan1993, Bertelli1994} to describe stars with varying metallicities. Both the supernova ``blastwave'' radius and supernova (Ia and II) prescriptions are as described in detail by \cite{Stinson2006}. A low-temperature extension to the cooling curve is used to trace metals \citep{Bromm2001}. Simulated galaxies are shown to conform with the observed Tully-Fisher relation \citep{Governato2009}, the size-luminosity relation \citep{Brooks2011}, and the mass-metallicity relation \citep{Brooks2007}, in addition to having realistic matter distributions and baryon fractions \citep{Governato2009a,Guedes2011}. Parameter and resolution choices described above allow the galaxies to adhere to the stellar-mass-halo-mass relation at z=0 and maintain a realistic period of star formation \citep{Moster2010,Munshi2013,Brooks2007,Maiolino2008}. Given the strict adherence of these simulated galaxies with observations, we are confident that they reasonably represent growth in the galaxy and its SMBH. We exclude AGN feedback in these small to moderate mass-galaxies; however, we have determined that has little effect on the global properties of the galaxy and SMBH.
 
Since, there are uncertainties in the formation of black holes ``seeds,'' we implement a BH seeding method that is broadly consistent with several theories of direct collapse black holes \citep{Couchman1986, Abel2002, Bromm2004} and Population III stellar remnants \citep{Loeb1994, Eisenstein1995, Koushiappas2004, Begelman2006, Lodato2006}. While this method allows the BH formation process to remain physically motivated, BH seeds form if their parent gas particle match the criteria required for star formation and also maintain zero metallicity (2.5 amu cm$^{-3}$; T \textless 10$^4$ K; Z = 0) \citep{Stinson2006}. A probability of $\chi_{\rm seed}$ $\sim$ 0.01 is applied to determine whether a gas particle (with the above specifications) will become a BH seed with a mass of M$_{\rm BH}$ = 2.28 $\times$ 10$^5$ M$_{\odot}$, the same mass as its parent gas particle. This probability was chosen to match the predicted occupation fraction of BH seeds at z $\sim$ 3 \citep{Volonteri2008}.
%This circumvents the Soltan argument wherein mergers alone are not the primary feeders of BHs, allowing for accretion growth to dominate \citep{Soltan1982}.  
%BH seed formation ends at z $\sim$ 3.5 due to the dilution of pristine gas by efficient metal diffusion expected after the first population of supernovae. 
The requirement that BH seeds must form from zero metallicity gas particles also causes BH formation to be constrained in areas of early star formation bursts, where the earliest and most massive halos are expected to form in the simulation. BH formation is dependent only on local environment, neglecting any large-scale properties of the host halo. Black holes are not fixed within the center of their host, allowing them to be dynamically affected by mergers and other perturbations within the galaxy. Nevertheless, BHs remain near their host centers by choosing dark matter particle masses in high-resolution regions to be on the same order as gas particle masses, minimizing two-body interactions \citep{Bellovary2011}.

Black hole mergers occur when they are separated by less than twice the softening length, and must be bound or satisfy $(1/2) \delta v^2 < \delta a \cdot \delta r$,  where $\delta v$ and $\delta a$ are the velocity and acceleration differences between the two black holes and $\delta r$ is the distance separating them. In addition to gaining mass via merger, black holes gain mass through Bondi-Hoyle gas accretion:
\begin{equation}
\dot{M} = \frac{4 \pi \alpha G^2 M^{2}_{\rm BH} \rho}{(c^{2}_{s} + v^2)^{3/2}},
\end{equation}
where $\alpha$ is a constant equal to 1, $\rho$ is the density of the surrounding gas, $c_s$ is the sound speed, and $v$ is the black hole's relative velocity to the gas. Feedback is applied to the 32 closest particles with an energy boost determined by the accreted mass as follows: $\dot{E}$ = $\epsilon _{r}$$\epsilon_{f}$$\dot{M}$$c^2$ where $\dot{M}$ is the accreted mass, and $\epsilon _r = 0.1$ and $\epsilon _f = 0.03$ are assumed for the radiative efficiency and feedback efficiency, respectively. This energy is distributed as thermal energy to the 32 nearest particles via a kernal probability function. Though other groups use a higher value for this efficiency, $\epsilon _f = 0.05$ \citep{Sijacki2007,DiMatteo2008} We find that $\epsilon_f = 0.03$ in our code produces MBHs in better agreement with MBH-host galaxy scaling relations. However, as our main concern is in the relative proportion of gas in various phases, our results are not sensitive to our choices of $\epsilon _{r}$ or $\epsilon_{f}$.
 
\section{\normalsize Low Resolution GASOLINE Simulations}

Our low resolution simulations were first run using the smoothed particle hydrodynamics (SPH) N-body tree code Gasoline \citep{Wadsley2004}, with a force resolution of 320 pc and the gas, dark matter, and star particle masses of 2 $\times$ 10$^5$ M$_{\odot}$, 3 $\times$ 10$^5$ M$_{\odot}$, and 7 $\times$ 10$^4$ M$_{\odot}$, respectively. Both N-body+SPH volume renormalized simulations of h258 and h277 have been previously examined in other publications. \cite{Governato2009} studied the regrowth of the large disk structure in h258 by z=0 when such a recent merger affects the galaxy. This code includes a force resolution of 260 comoving pc, dark matter (DM) particles with masses of $1.3 \times 10^5 M_{\odot}$, and initial gas particle masses of $9 \times 10^4 M_{\odot}$.  
	
\section{\normalsize High Resolution CHANGA Simulations}

We used Charm N-body GrAvity solver, ChaNGa, a high resolution version of the same volume. The results of this paper, both from Gasoline and ChaNGa, utilize the same initial conditions but the latter is simulated at a higher resolution allowing the following new physics to be included: metal line cooling \citep{Christensen2012,Shen2010}, gas-phase and dust grain H$_2$ formation \citep{Christensen2014}, star formation dependence on H$_2$, and H$_2$ destruction from Lyman-Werner radiation. Other changes include a spline force softening lengths of 174 pc and initial gas particle masses of $2.7 \times 10^4 M_{\odot}$. Star particles are created with 30\% of their parent gas particle mass; therefore, a maximum initial mass of 8100 $M_{\odot}$. Each galaxy contains about 5 million DM particles inside the virial radius at z=0 and over 14 million DM, star, and gas particles total. The resolution of both force and mass in these simulations is comparable to the ``Eris'' simulation which has one of the highest resolutions for an N-body+SPH cosmological simulation of a Milky Way-mass galaxy so far produced.  At z=9, a uniform UV background is applied to simulate the cosmic reionization energy in a variation of \cite{Haardt2012}.

In this paper, we demonstrate that the results from this high resolution ChaNGa simulation has results which confirm those found by the low resolution Gasoline version of the same galaxy simulation. This secondary analysis allows us to feel confident in our results as we describe the varying origins of fueling gas accreting onto the simulated Milky Way-mass galaxy, h258.
% *** END OF SIMULATION PARAMETERS ***


% *** METHODOLOGY ***
\chapter{\normalsize METHODOLOGY}
\thispagestyle{empty}

The Amiga Halo Finder identifies all of the halos in the simulations by identifying the virial radii of the halos based on an overdensity using the criterion for a flat universe \citep{Knebe2001,Knollmann2009,Gill2004}. We select the primary halo by determining the most massive galaxy at z=0 and locating the central SMBH. For our Gasoline simulations, both the primary halo in h258 and h277 had final masses on the order of $10^{12} M_{\odot}$ and a formation redshift of z$\sim$4.  

%EDIT: Include specs: mass, formation z, N (just for primary halos? Ask Kelly what she means.)

In this analysis, we trace gas particles from their origin through their journey into the galaxy and SMBH to determine their position within the SMBH or its host \citep{Brooks2009} . We then trace the SMBH back through time, determining its host halo in each snapshot. As we discussed in the introduction, there are a variety of fueling mechanisms thought to grow SMBH. Gas particles are initially traced prior to their accretion onto the main halo to determine whether they occupied a different halo at their formation. The particles are then classified by their method of entrance into the primary halo. If the gas particles existed in the primary halo at the first time step, they are classified as ``early'' gas. Gas that belonged to a different halo than the primary prior to accretion is classified as ``clumpy'' and enters the primary halo through mergers. All other gas is classified as entering the halo via ``smooth'' accretion, and is then subdivided into two categories: ``cold'' and ``shocked.'' In the cases where a shock does not develop, unshocked gas will usually flow into the halo via large-scale, dark matter filaments. (Bellovary 2013) It's possible for the filaments to also be dense enough to pierce an already developed shock allowing for unshocked gas to funnel into the galaxy core where it can accrete onto the SMBH without being heated to the virial temperature.

As previously described, if the galaxy halo is around or greater than 10$^{11}$ M$_{\odot} $, the gas will shock. We determine shocked particles through an increase in entropy and temperature using the following criteria:
\begin{equation}
T_{\rm shock} \geq 3/8 T_{\rm vir}
\end{equation}
where T$_{\rm vir}$ is the virial temperature of the halo and T$_{\rm shock}$ is the temperature of the gas particle, and 
\begin{equation}
\Delta S \geq S_{\rm shock} - S_0
\end{equation}
where S$_0$ is the initial entropy of the gas particle and 
\begin{equation}
S_{\rm shock} = log_{10}[3/8 T_{\rm vir}^{1.5}/4 \rho_0]
\end{equation}
where $\rho_0$ is the gas density prior to experiencing the shock. The galaxies in this study reach final masses on the order of 10$^{12}$ M$_{\odot} $, therefore, shocked gas begins to enter the galaxy by the later timesteps of our simulation where it has reached this critical mass. 

Once all the gas particles have been individually categorized, we can use these labels to determine the methods of accretion onto these Milky Way-size galaxies. With the information about which particles are accreting onto the SMBH (from the gas reservoir of particles that have been accreted by the galaxy), we can better compare the methods feeding both the galaxy and its SMBH. We can then determine whether there exists a method more efficiently building up these kinds of galaxies and their SMBH.
% *** END OF METHODOLOGY *** 


% *** RESULTS ***
\chapter{\normalsize RESULTS}
\thispagestyle{empty}
From the initial analysis of the low resolution Gasoline simulations, we have determined that the merger histories of these two galaxies has a pronounced effect on the resulting accretion of gas into the SMBH and its host. 

\section{Gasoline - Low Resolution h258: Resolution Test}
	
The galaxy h258 is characterized by an active merger history, including a major merger (q=0.9, with an incoming gas mass of 1.26 $\times$ 10$^10$ $M_{\odot}$) at z=1 (Figure \ref{h258face}). The merger can be seen increasing the cumulative black hole mass (Figure \ref{h258allmassgas}) as well as the gas fractions within both the SMBH and galaxy (Figure \ref{h258numfrac}). Figure \ref{h258allmassgas} shows the cumulative SMBH mass in h258 as a function of time (lower axis) and redshift (upper axis). The black dashed line indicates the total cumulative BH mass (including both mass from gas and BH mergers), while the black solid line indicates the total accreted gas mass. The blue dot-dashed line represents the gas mass accreted via unshocked gas, while the green solid line and red dashed line show the gas mass accreted through mergers and shocked gas, respectively. It is also worthwhile to point out that the largest part of the mass budget at high redshift is not gas at all, but other black holes that have merged with the SMBH seed. This has important implications for gravitational wave astronomy, increasing the event rate for SMBH assembly at high redshifts \citep{Holley-Bockelmann2010}. Aside from this early BH assembly, the largest gain in SMBH mass comes from gas associated with the major merger at z $\sim$ 1. Figure \ref{h258numfrac} shows the fractions of unshocked and clumpy gas compared to all accreted gas as a function of redshift. It confirms that unshocked gas makes up the majority of gas entering the galaxy at early times. Nevertheless, around the time of the merger (z $\sim$ 1) a marked change occurs in the gas fractions and clumpy gas becomes the main component of total gas in the BH.

\begin{figure}
\centerline{\resizebox{0.75\hsize}{!}{\includegraphics[angle=0]{h258_allmassgas_final}}}
\caption[GASOLINE h258 SMBH Cumulative Mass]{The cumulative mass of h258's SMBH as a function of time and redshift. The black dashed line indicates the total BH mass. The black solid line indicates the total gas mass. The blue dot-dashed line indicates the gas mass accreted via unshocked gas, while the green solid and red dashed lines indicate the mass accreted through mergers and shocked gas, respectively.}
\label{h258allmassgas} 
\end{figure}

Figure \ref{h258stackfrac} depicts the fractions of total gas in the galaxy (a) and the SMBH (b) at z=0, again differentiated by gas origin. Blue, green, and red distinguish cold, clumpy, and shocked, respectively. Yellow indicates gas that existed within the main halo upon formation. The galaxy has a mass nearly half comprised of gas entering the galaxy through unshocked, smooth accretion (48 \%), with 39 \% of the gas entering through mergers. The smallest fractions of the total gas are comprised of shocked gas (9 \%) and ``early''gas that existed at formation makes up the final, smallest fraction (4 \%). Unlike the galaxy, nearly 65 \% of the gas accreted by the central SMBH was gas accreted via mergers, while only a quarter (26 \%) is comprised of unshocked, smoothly accreted gas. Shocked gas makes up the last 9 \% of total gas entering the SMBH, along with a negligible ($<$ 1\%) amount of ``early'' gas. \textbf{It is evident then that the SMBH more readily accretes gas gained through mergers.} This result is contrary to \cite{Bellovary2013} which found that the fractions of gas comprising the SMBH and its host were nearly the same. Through our results, we find that lower mass galaxies can readily employ the physical effects of mergers to feed their SMBH.

\begin{figure}[h]
\centerline{\resizebox{0.75\hsize}{!}{\includegraphics[angle=0]{h258numfraction_brightgreen}}}
\caption[GASOLINE h258 Galaxy and SMBH Gas Fractions Across Time]{Gas fraction across redshift for the h258 galaxy (solid lines) and central BH (dashed lines). Green lines signify gas fractions accreted via mergers and blue lines designate gas accreted via unshocked gas filaments.}
\label{h258numfrac} 
\end{figure}

To better understand the apparent preference for merger-accreted gas, we examine the angular momentum of each type of gas at the moment it enters the galaxy Figure \ref{h258angmom}) shows that gas entering the SMBH (dashed lines) has an overall lower angular momentum than gas entering galaxy (solid lines), as expected; this statement is true regardless of the gas state. 

Figure \ref{h258angmom_merger} explicitly shows that at the time of the merger, the lowest angular momentum gas entering the galaxy (solid lines) is merger-driven (green, solid line). This is reflected in the SMBH (green, dashed line) as nearly all the clumpy gas at this time funnels directly into the SMBH.

%\textit{NOTE: Need to add a figure that clearly supports this argument. Possibly a series of angular momentum plots at the times of mergers. Or comparisons between cumulative distributions at time with and without mergers.}

\begin{figure}[h]
\centerline{\resizebox{0.75\hsize}{!}{\includegraphics[angle=0]{h258_stackbarfractions}}}
\caption[GASOLINE h258 Galaxy and SMBH Gas Fractions at z=0]{Gas fractions of accreted gas in h258 by the main halo (left) and the SMBH (right), distinguished by type. Blue, green, and red distinguish gas gained through unshocked gas, gained through mergers, and gas shocked upon entry, respectively. Yellow indicates gas that existed within the main halo upon formation; this ``early'' gas is negligible ($<$ 1 $\%$) within the SMBH.}
\label{h258stackfrac} 
\end{figure}

\begin{figure}[p]
\centerline{\resizebox{0.75\hsize}{!}{\includegraphics[angle=0]{h258_angmom_cumudist}}}
\caption[GASOLINE h258 Cumulative Angular Momentum Distribution]{ Cumulative distribution of angular momentum of gas accreted onto h258.  Gas accreted onto the main halo (solid lines) and central black hole (dashed lines). The green, blue, and red lines indicate clumpy, unshocked, and shocked gas, respectively.}
\label{h258angmom} 

\centerline{\resizebox{0.75\hsize}{!}{\includegraphics[angle=0]{h258_am_tsnew_228}}}
\caption[GASOLINE h258 Merger Angular Momentum Distribution]{ Cumulative distribution of angular momentum of the gas particles accreted onto h258 at the time of the major merger (z $\sim$ 1). There are about 95,000 gas particles accreted at this timestep. Gas accreted onto the main halo and central black hole are distinguished by solid and dashed lines. The green, blue, and red lines indicate clumpy, unshocked, and shocked gas, respectively.}
\label{h258angmom_merger} 
\end{figure}


\section{ChaNGa - High Resolution h258}

Just as we found in the same galaxy, h258, of the lower resolution simulation, the high resolution h258 galaxy sees a distinct preference for accreting gas that has been gained through the many mergers in its history. In this version, h258 is characterized by two major mergers; the first occurs at z $\sim$ 1.8 and the second at z $\sim$ 1.2. Unlike its predecessor, this h258 has a smoother history of overall gas accretion (Figure \ref{hrh258allmassgas}; Linestyles as in Figure \ref{h258allmassgas}). However, the transitions between when smooth, unshocked accretion and mergers dominate is clearly distinguished. While clumpy gas (green) dominates at the earliest time, unshocked gas (blue) overtakes it for a short time ($\sim$ 2 Gyr) before clumpy gas once again dominates by z $\sim$ 1.5. 

\begin{figure} [h]
\centerline{\resizebox{0.75\hsize}{!}{\includegraphics[angle=0]{hrh258_massgastime}}}
\caption[ChaNGa h258 SMBH Cumulative Mass]{The central BH’s cumulative mass as a function of time and redshift. The black dashed line indicates the total cumulative BH mass. The black solid line indicates the total gas mass. The blue dot-dashed line indicates the gas mass accreted via unshocked gas. The green solid line indicates the gas mass accreted through mergers. The red dashed line indicates gas mass that was shocked upon entry into the halo.}
\label{hrh258allmassgas} 
\end{figure}

This low redshift transition to a clumpy gas preference results in the large fraction of clumpy gas seen in the SMBH (Figure \ref{hrh258stackfrac}; Colors as in Figure \ref{h258stackfrac}). We can see from the figure that comparable to the Gasoline h258 version, the amount of unshocked, cold gas in the SMBH (24\%) is less than that entering the galaxy (36\%). The shocked gas, however, sees a sharper decline in the amount of shocked gas that enters the SMBH ($\sim$2\%) than in the host (8\%). This differs from the Gasoline h258 results and may be attributed to the higher resolution of the second iteration. Figure \ref{hrh258stackfrac} also explicitly shows how the fraction of clumpy gas accreted by the SMBH (74\%) is significantly larger than the fraction of clumpy gas accreted by the host galaxy (56\%).  \textbf{This result, which repeats the analysis with an improved and higher resolution simulation, confirms the initial study's findings!} This confirmation makes us confident that these results are well founded. 

\begin{figure}[h]
\centerline{\resizebox{0.75\hsize}{!}{\includegraphics[angle=0]{hrh258_stackbarfractions}}}
\caption[ChaNGa h258 Galaxy and SMBH Gas Fractions at z=0]{Gas fractions of the gas particles accreted by the high resolution ChaNGA h258 by the main halo (left) and the SMBH (right), distinguished by type. Blue, green, and red distinguish gas gained through unshocked gas, gained through mergers, and gas shocked upon entry, respectively. Yellow indicates gas that existed within the main halo upon formation; this ``early'' gas is negligible ($<$ 1 \%) within the SMBH.}
\label{hrh258stackfrac} 
\end{figure}

\begin{figure}[h]
\centerline{\resizebox{0.75\hsize}{!}{\includegraphics[angle=0]{hrh258_am_cumu_4096}}}
\caption[ChaNGa h258 Cumulative Angular Momentum Distribution]{ Cumulative distribution of angular momentum of the gas particles accreted onto the high resolution ChaNGa h258.  Gas particles accreted onto the main halo (solid lines) and central black hole (dashed lines). The green, blue, and red lines indicate clumpy, unshocked, and shocked gas, respectively.}
\label{hrh258angmom} 
\end{figure}

In keeping with the analysis of the Gasoline galaxies, we followed up the study of the galaxy and SMBH gas fractions by examining the angular momentum of the accreted gas particles. Figure \ref{hrh258angmom} (Colors as in Figure \ref{h258angmom}.) shows a cumulative distribution of the gas particles' angular momentum as they enter the halo, distinguished between gas that ends up in the SMBH (dashed line) or the galaxy (solid lines) at z=0. This plot is consistent with Figure \ref{h258angmom}, as the angular momentum of gas entering the SMBH is lower when comparing between types of gas. In the ChaNGA h258 case, however, we see that the lowest angular momentum gas is comprised of both clumpy and unshocked gas, unlike in the Gasoline version which clearly showed that unshocked gas entered the SMBH with the lowest angular momentum. We attribute this variation to the subtle difference between the histories of the two simulations, including the two mergers which characterize the higher resolution version. This can be seen if we examine Figure \ref{hrh258angmom_merger2} which shows the cumulative distribution of the angular momentum of the incoming gas particles at the time of the second merger. (Colors and linestyles as in Figure \ref{h258angmom_merger}.) Figure \ref{hrh258angmom_merger2} explicitly shows that the gas ending up in the SMBH enters with the lowest angular momentum. This result is similar to what we found in the Gasoline h258 simulation, however, in the previous case it appeared the all the low angular momentum gas entering at the merger timestep ended up in the galaxy. Yet the distinction remains as there appears to be an influx of gas with low angular momentum feeding the black hole at this timestep that is not associated with the merger occurring at z $\sim$ 1.2. This may be attributed to the prior merger at z $\sim$ 1.8 (Figure \ref{hrh258angmom_merger1}) during and after which a large influx of low angular momentum gas enters the galaxy and SMBH. (Color and linestyles as in Figure \ref{hrh258angmom_merger2}) 


\begin{figure}
\centerline{\resizebox{0.75\hsize}{!}{\includegraphics[angle=0]{hrh258_am_tsnew_1128}}}
\caption[ChaNGa h258 First Merger Angular Momentum Distribution]{ Cumulative distribution of angular momentum of the gas particles accreted onto the high resolution ChaNGA h258 galaxy at the time of the first major merger (z $\sim$ 1.8). There are about 900,000 gas particles accreted at this timestep, which gas particles accreted onto the main halo and central black hole are distinguished by solid and dashed lines. The green, blue, and red lines indicate clumpy, unshocked, and shocked gas, respectively.}
\label{hrh258angmom_merger1} 

\centerline{\resizebox{0.75\hsize}{!}{\includegraphics[angle=0]{hrh258_am_tsnew_1584}}}
\caption[ChaNGa h258 Second Merger Angular Momentum Distribution]{ Cumulative distribution of angular momentum of the gas particles accreted onto the high resolution ChaNGA h258 galaxy at the time of the second major merger (z $\sim$ 1.2). There are about 450,000 gas particles accreted at this timestep, which gas particles accreted onto the main halo and central black hole are distinguished by solid and dashed lines. The green, blue, and red lines indicate clumpy, unshocked, and shocked gas, respectively.}
\label{hrh258angmom_merger2} 
\end{figure}

%[Talk to Jillian and Kelly about this. Is this a reasonable explanation?]

\section{Gasoline - Low Resolution h277: Resolution Test}
	
Our quiescent galaxy, h277, was characterized by few mergers in its past, with final significant merger (q=0.14) occurring around z $\sim$ 3 (Figure \ref{h277face}). This merger is visible in Figure \ref{h277allmassgas} as a significant spike in the clumpy gas (green, solid line); however, the overall accretion of the SMBH is more subtle than the influx of gas in h258. (Linestyles as in Figure \ref{h258allmassgas}.) In h277, gas in the SMBH is gained at a more steady rate and at fairly high redshifts comprises a larger portion of the BH's total mass than seen in h258. The gas fractions (Figure \ref{h277numfrac}) in h277 (solid lines) and SMBH (dashed lines) are also markedly different. While it appears neither clumpy or unshocked gas (green and blue, respectively) dominates at high redshifts in the galaxy, merger gas dominates in the SMBH at the time of the last significant merger (z $\sim$ 3, with a gas mass of 4.43 $\times$ 10$^9$ M$_{\odot}$). Again, some physical process allows merger gas to readily fuel black hole growth. Though, it is important to note that by z $\sim$ 0, smoothly accreted, unshocked gas again dominates in both the SMBH and host galaxy. As in our more active galaxy, it appears that the role of clumpy gas as it fuels the SMBH becomes important when mergers affect the galaxy; however, since h277 is quiescent, smooth accretion inevitably dominates the total gas reservoir.

\begin{figure}[h]
\centerline{\resizebox{0.75\hsize}{!}{\includegraphics[angle=0]{h277_allmassgas_final}}}
\caption[GASOLINE h277 SMBH Cumulative Mass]{The cumulative mass of h277's SMBH as a function of time and redshift. The black dashed line indicates the total cumulative BH mass. The black solid line indicates the total gas mass. The blue dot-dashed line indicates the gas mass accreted via unshocked gas. The green solid line indicates the gas mass accreted through mergers. The red dashed line indicates gas mass that was shocked upon entry into the halo.}
\label{h277allmassgas} 
\end{figure}

This result can be more clearly seen in Figure \ref{h277stackfrac} which shows the z $\sim$ 0 gas fractions of the galaxy and SMBH. (Colors as in \ref{h258stackfrac}.) Smoothly accreted, unshocked gas represents about half of the total gas in both the SMBH and the galaxy. Clumpy gas also maintains about 30\% of both, while an increase in shocked gas occurs, from 13\% in the galaxy to 22\% in the SMBH; however, we argue this may be due to contamination in the simulation. Gasoline stops tracing shocked gas near the galaxy as shocks from supernovae can contaminate those results. We note that if gas is misclassified as shocked rather than unshocked, this may explain the apparent preference for shocked gas in the SMBH.
%We stop tracing shocked gas near the galaxy (~1/10th of the virial radius) bc shocks from SN can contaminate those results
%if gas is misclassified as shocked (should be cold) SMBH probably accretes the same amount of shocked as gxy, but if some of the cold gas is 
%misclassified as shocked, that may explain why we see an increase shocked in the SMBH.
%; however, we argue this may be due to supernova contamination in the simulation. 
Therefore, by present day, we argue that h277 finds no strong preference for the SMBH to prefer any one kind of gas in its accretion. Like the SMBHs in the study of \cite{Bellovary2013}, nearly the same fractions of gas entering the halo would then be accreted onto the SMBH (with the exception of shocked gas). Considering the early end to significant merger affects in h277 and the continued accumulation of smooth accretion and minor mergers, we determine that without some physical driver, like the late major mergers of h258, clumpy gas is unlikely to dominate the gas accreted into the SMBH. 

\begin{figure}[h]
\centerline{\resizebox{0.75\hsize}{!}{\includegraphics[angle=0]{h277numfraction_brightgreen}}}
\caption[GASOLINE h258 Galaxy and SMBH Gas Fractions Across Time]{Gas fraction across redshift for galaxy (solid lines) and central BH (dashed lines). Green lines signify gas fractions accreted via mergers and blue lines designate gas accreted via unshocked gas filaments.}
\label{h277numfrac} 
\end{figure}

\begin{figure}[h]
\centerline{\resizebox{0.75\hsize}{!}{\includegraphics[angle=0]{h277_stackbarfractions}}}
\caption[GASOLINE h258 Galaxy and SMBH Gas Fractions at z=0]{Gas fractions of the gas particles accreted in h258 by the main halo (left) and the SMBH (right), distinguished by type. Blue, green, and red distinguish gas gained through unshocked gas, gained through mergers, and gas shocked upon entry, respectively. Yellow indicates gas that existed within the main halo upon formation; this ``early'' gas is negligible ($<$ 1 \%) within the SMBH.}
\label{h277stackfrac} 
\end{figure}

As in the previous case, we examined the angular momentum of gas accreted by both the SMBH and the host galaxy halo. (Figure \ref{h277angmom}; Linestyles as in \ref{h258angmom}) Like before, it is immediately apparent that the clumpy gas accreted by both the black hole and its host has a lower angular momentum overall than the smoothly accreted gas; however, unlike we found in h258, the SMBH does not accrete the lower angular momentum materialå. This may be due to the gas entering the galaxy primarily through minor mergers after z $\sim$ 3 and therefore losing the physical driver of a major merger at later times. Figure \ref{h277angmom_merger} shows the very slight change towards lower angular momentum in the clumpy gas during the merger event. 

% IS THIS WHAT IT MEANS? This means that even though some of the gas in the host halo had lower angular momentum, that gas didn't find its way into the central region to accrete onto the BH. This may again imply that some physical phenomena is required to drive those gases towards the center. 


\begin{figure}
\centerline{\resizebox{0.75\hsize}{!}{\includegraphics[angle=0]{h277_angmom_cumudist}}}
\caption[GASOLINE h258 Cumulative Angular Momentum Distribution]{ Cumulative distribution of angular momentum of the gas particles accreted onto h277.  Gas particles accreted onto the main halo (solid lines) and central black hole (dashed lines). The green, blue, and red lines indicate clumpy, unshocked, and shocked gas, respectively.}
\label{h277angmom} 

\centerline{\resizebox{0.75\hsize}{!}{\includegraphics[angle=0]{h277_am_tsnew_096}}}
\caption[GASOLINE h258 Merger Angular Momentum Distribution]{ Cumulative distribution of angular momentum of the gas particles accreted onto h277 at the time of the last major merger (z $\sim$ 3). There are about 50,000 gas particles accreted at this timestep. Gas accreted onto the main halo and central black hole are distinguished by solid and dashed lines. The green, blue, and red lines indicate clumpy, unshocked, and shocked gas, respectively.}
\label{h277angmom_merger} 
\end{figure}

As h277 and h258 are similar in size and appearance at z=0, these differences indicate that varying galaxy assembly may play an important role in the evolution of these types of galaxies and the present day composition of their central SMBHs. We also note that while mergers may not be the only physical mechanisms that funnel gas into a SMBH, they play an important role when present.


% *** END OF RESULTS ***


% *** DISCUSSION ***
\chapter{\normalsize DISCUSSION}
\thispagestyle{empty}
This study examines the gas accretion onto two fully cosmological simulations of Milky Way-size galaxies to redshift z $=$ 0, one with a quiescent merger history and one with major mergers characterizing its past. A second iteration of the latter, simulated with higher resolution and with improved physics, is also examined. In each case, we trace the gas into the SMBHs at their centers and differentiate the gas accreted onto the galaxy and SMBHs by origin. Gas gained through mergers is classified as ``clumpy'' gas and smoothly accreted gas is separated into ``shocked'' and ``unshocked'' categories. Our goal is to determine what types of gas are primarily feeding the SMBH and the galaxies of this class, and to determine what effects the merger histories of the galaxies may have on these processes.

When we examine the accretion of the galaxy with a quiescent history, we note that smooth accretion is the dominant feeding mechanism for both the galaxy and its central SMBH. We find that the fractions of each type of gas comprising the overall mass of the SMBH are comparable to the fractions of the gases making up its host. This result is consistent with the results of \cite{Bellovary2013} which similarly analyzed high mass, high redshift galaxies and found the gas composition of the SMBHs mirror their host.

Contrary to these previous results, when we examined the galaxy with an active merger history, we determined that the SMBH at the center more readily accretes gas gained through mergers. \textbf{In both the low and high resolution cases, we see a significant increase in the clumpy gas accreted by the SMBH compared to its host.} 

The angular momentum of the accreted gas as it enters the galaxy sheds some light on the mechanism driving this preferentially accreted clumpy gas. Smoothly accreted gas, which enters with galaxy with a wide range of angular momentum, may become additionally torqued by the already existing disk. Meanwhile, gas entering through mergers is restricted by the direction of its entry and allows for a small range of low angular momentum gas to enter the main halo. This gives clumpy gas the advantage of falling more readily to the center and accreting onto the SMBH. Considering all origins of gas, we see a clear distinction wherein lower angular momentum gas preferentially feeds the SMBH.

This however, does not remain true when we look again at the galaxy with a quiescent history. In this case, some of the gas entering the galaxy with the lowest angular momentum remains outside of the SMBH. This result implies that some of the physical processes affecting the galaxy with an active merger history accounts for the ready funneling of low angular momentum to the central region of the black hole. 

While the examination of these two extreme cases of galaxy merger history depict a class of galaxy with varying SMBH accretion methods, a further study of intermediate cases is required to begin understanding the broad spectrum of Milky Way-mass galaxy accretion. It is clear through this study that the presence (or absence) of major mergers can play an important role in the final compositions of central SMBHs, but the question of how important these mergers are remains to be seen.
% *** END OF DISCUSSION ***



\backmatter
\chapter{CODE}
\section{IDL Scripts for ChaNGa h258}
As the Gasoline versions of h258 and h277 had been previously studied and my research was an extension of \cite{Bellovary2013}, many of the reduction scripts for these galaxies mostly existed. I waw required to make sure the majority of the scripts would transfer their analysis correctly onto these two galaxies. These scripts, which are not included as they belong more directly to their primary author, are available at https://github.com/nnicolesanchez/mastersidlscripts along the full author credits.

Therefore, for the ChaNGa h258 galaxy, a similar analysis needed to be done and comparable scripts needed to be written to examine this high resolution version.

\subsection{highres\_bhtrace\_latez.pro}
\begin{verbatim}
;# This program determines which bhs merged into our main halo prior to
;# timestep 336. We assume we already know which black hole is in our main halo
;# after timestep 336 (360 and above). For the case of CHANGA h258,
;# this bhiord is  BHIORD 43553282 (B82) so we search for which 
;# BHs merged in early z (before timestep 336) to form B82.

;# The steps we take:
;# Read in fits file with information about galaxy sim
;#     Currently examining high rez, Changa simulation of h258
;# Parse out array with B82s info
;#     file.bhiords    (bh id)
;#     file.afac       (a factor which calculates z)
;#     mbhf            (final black hole mass; we could also use mbhi)
;#     mbhi            (same as mbhf except shifted forward one step in array)
;#                     (ie mbhf[4] = mbhi[5])
;# Calculate difference in mass between each time step
;# Create array with these values that are greater than the bimodal break in a
;#     distribution of mbhf; Also get times
;# Determine how bhs and their info look once the bh has disappeared/merged
;# Parse out info for each of the bhs that have disappeared
;# Find the mass at which these bhs have disappeared
;# Compare to the array of times and masses
;# Determine which galaxies merged into the main halo
;# Boom.

;# Vanderbilt Univ.  -- VPAC40: /home/sanchenn/IDL/highres_bhtrace_latez.pro
;# N. Nicole Sanchez -- Last edit: January 23, 2016
pro highres_bhtrace_latez,bhofinterest,bimodalcutoff

;# Input Parameters
if keyword_set(bhofinterest) then begin
   bhofinterest=bhofinterest  
endif else begin
   print,'What BH is most central/massive?' 
   stop
endelse
if keyword_set(bimodalcutoff) then begin
   bimodalcutoff = bimodalcutoff 
endif else begin
   bimodalcutoff = 1.0d4        ;For h258, & should work usually
endelse
munit      = 1.84793d16         ; M_sun
loadct,4

;# Read in fits files
file       = mrdfits('../BHAccLog.fits',1)
;help,file

;# Calculate z from a factor to sort by
z          = (1. /file.afac) - 1.
print,'Starting redshift',z[0]

;# Set Main BH (for high res h258: B82)
MBH_ind    = where(file.bhiord eq bhofinterest) 
MBH_bhiord = file.bhiord[MBH_ind]
MBH_mbhi   = file.mbhi[MBH_ind] * munit   ; M_sun
MBH_mbhf   = file.mbhf[MBH_ind] * munit   ; M_sun
MBH_dmacc  = file.dmacc[MBH_ind] * munit  ; M_sun
MBH_mgi    = file.mgi[MBH_ind] * munit    ; M_sun
MBH_z      = z[MBH_ind]
;# NOTE: dmacc is change in gas mass from accr, does not include 
;# gas accr from BH; will not use in rest of this script
;# (Included above only to remind you, Nicole)

;# THERE IS WEIRDNESS: Sometimes mgi = 0, these are bad lines
;# Remove these then redefine all lines
goodlines  = where(MBH_mgi ne 0)
MBH_bhiord = MBH_bhiord[goodlines]
MBH_mbhf   = MBH_mbhf[goodlines]
MBH_mbhi   = MBH_mbhi[goodlines]
MBH_z      = MBH_z[goodlines]

;# Sort in time
z_ind     = sort(MBH_z)
MBH_mbhf  = MBH_mbhf[z_ind]
MBH_mbhi  = MBH_mbhi[z_ind]
MBH_z     = MBH_z[z_ind]
nmbhf = n_elements(MBH_mbhf)
print,'Number of BH Masses',nmbhf

;# Calculate deltambhf and find bimodal cutoff
MBH_deltambhf = fltarr(n_elements(MBH_mbhf))
for i=1,n_elements(MBH_mbhf)-1 do begin
   MBH_deltambhf[i] = MBH_mbhf[i-1] - MBH_mbhf[i]
endfor

;# Plot distribution of ∆mbhf to determine where your mass cutoff is
plothist,alog10(MBH_deltambhf[where(MBH_deltambhf gt 0.0)]),/ylog,xtitle='Log o
f Changes in Final BH Masses in M_sun',yrange=[0.1,100000]
stop

;# Plot mbhf across time to see all mergers (big steps)
plot,MBH_z,MBH_mbhf,xtitle='Redshift',ytitle='Log of Central BH Mass in Msun',/
ylog,xrange=[25,0]
stop

;# Create array with deltambhf values greater than cutoff value
;# Cut off value for h258 = 10^4
MBH_merger_masses = [0]
MBH_merger_times  = [0]
MBH_mass_at_merger = [0]
;print,1.0d4
for i=0,n_elements(MBH_deltambhf)-1 do begin
   if (MBH_deltambhf[i] gt bimodalcutoff) then begin
      new_dmacc = MBH_dmacc[i]
      new_deltambhf = MBH_deltambhf[i]
      new_z = MBH_z[i]
      new_mbhf = MBH_mbhf[i]
      MBH_merger_masses = [MBH_merger_masses,new_deltambhf] 
      MBH_merger_times   = [MBH_merger_times,new_z]
      MBH_mass_at_merger = [MBH_mass_at_merger,new_mbhf]
      ;print,new_deltambhf
      ;print,new_z
      ;print,new_mbhf
   endif
endfor

;print,MBH_mergers[0]
;print,n_elements(MBH_mergers)
;# NOTE: I'm doing a minus 2 at the end of these array redefinitions
;# because there is some sort of bug in the highest redshift (large
;# z) values.
MBH_merger_masses  = MBH_merger_masses[1:n_elements(B52_merger_masses)-1]
MBH_merger_times   = MBH_merger_times[1:n_elements(B52_merger_times)-1]
MBH_mass_at_merger = MBH_mass_at_merger[1:n_elements(B52_mass_at_merger)-1]
print,MBH_merger_masses
print,'Number of main mergers',n_elements(MBH_merger_masses)
;stop

;# Figure out a way to determine when a black hole disappears
uniq_ind = uniq(file.bhiord)
uniq_ids = file.bhiord[UNIQ(file.bhiord, SORT(file.bhiord))]
;print,uniq_ids[0:100]
print,'Number of unique BH ids',n_elements(uniq_ids)

bh_maybe_id   = [0]
bh_maybe_mass = [0]
bh_maybe_time = [0]
print,'Number of Timesteps in Main BH',nmbhf
for i=0,n_elements(uniq_ids)-1 do begin
   bh_ind  = where(file.bhiord eq uniq_ids[i])
   bh_ids  = file.bhiord[bh_ind]
   ;print,'These should all be the same',bh_ids
   bh_mbhf = file.mbhf[bh_ind] * munit   ; M_sun
   bh_z    = z[bh_ind]
   z_ind   = sort(bh_z)
   bh_mbhf = bh_mbhf[z_ind]
   bh_z   = bh_z[z_ind]
   nbh     = n_elements(bh_ind)
   ;print,'Number of times BH appears in',nbh

   if (bh_z[0] ne MBH_z[0]) then begin
      new_id   = bh_ids[0]
      new_mass = bh_mbhf[0]
      new_time = bh_z[0]
      
      bh_maybe_id = [bh_maybe_id, new_id]
      bh_maybe_mass = [bh_maybe_mass, new_mass]
      bh_maybe_time = [bh_maybe_time, new_time]
   endif
endfor

bh_maybe_id = bh_maybe_id[1:n_elements(bh_maybe_id)-1]
bh_maybe_mass = bh_maybe_mass[1:n_elements(bh_maybe_mass)-1]
bh_maybe_time = bh_maybe_time[1:n_elements(bh_maybe_time)-1]
;#print,bh_maybe_id
;#stop


for i=0,n_elements(bh_maybe_mass)-1 do begin
   for j=0,n_elements(MBH_merger_masses)-1 do begin
;      if (bh_maybe_mass[i] ge B52_merger_masses[j]-3) and (bh_maybe_mass[i] le
 B52_merger_masses[j]+3) then begin
;         print,bh_maybe_id[i]
;         print,bh_maybe_mass[i]
;         print,bh_maybe_time[i]
;      endif
      if (bh_maybe_time[i] ge MBH_merger_times[j]-0.02) and (bh_maybe_time[i] l
e MBH_merger_times[j]+0.02) then begin
         print,'Disappearing BH id',bh_maybe_id[i]
         print,'Disappearing BH mass',bh_maybe_mass[i]
         print,'Disappearing BH redshift',bh_maybe_time[i]

         print,'Main BH mass at this time',MBH_mass_at_merger[j]

      endif
   endfor
endfor

;####### Random notes for Nicole 
;# Remember: for h258, we know the main halo before 144 (z=10.10)
;# SO, we only have to look at mergers before this to determine
;# who the progenitors are
;IDL> rtipsy,'h258.cosmo50cmb.3072gst10bwepsK1BHC52.000144',h,g,d,s
;% Compiled module: RTIPSY.
;% Compiled module: SWAP_ENDIAN.
;IDL> help,h
;** Structure <1ff7d88>, 6 tags, length=32, data length=28, refs=1:
;   TIME            DOUBLE         0.090039161
;   N               LONG          43560070
;   NDIM            LONG                 3
;   NGAS            LONG           9429025
;   NDARK           LONG          34123819
;   NSTAR           LONG              7226
;IDL> print,h.time
;     0.090039161
;IDL> a=h.time
;IDL> z          = (1. /a) - 1.
;IDL> print,z
;       10.106278

end
\end{verbatim}

\subsection{highres\_gasiords.pro}
\begin{verbatim}
;# This script is a redux of Jillian's gasiords.pro (find in
;# Vanderbilt VPAC40 /home/sanchenn/IDL/gasiords.pro which
;# works for low res, Gasoline simulation outputs) 
;# This script works for high res, CHANGA files instead. 

;# The steps we take:
;#    Pull out the information from BHAccLog.fits for gasiords.pro
;#       BHIORDS          (ids for each bh)
;#       BHMBHF           (mass of black hole at end of timestep;
;#                        'final') Not used in this script. Just for out.bh
;#       GASIORD          (id of eaten mass)
;#       DMQ/EATEN MASS   (eaten mass of gas only)
;#       TIME/Z           (time/redshift)
;#    Specifically, pull out info for bhs of interest (h258: B52&B62)
;#    Put in chronological order
;#    With merger time from highres_bhtrace_latez_better.pro,
;#       Concatenate in time order
;#    Run through Jillian's gasiords.pro prescription

;# NOTE: You will concatenate the info for B62 and B52 for 
;#       these arrays of data. See highres_bhtrace_latez_better.pro 
;#       for details on how central/massive BHs were determined

;# IMPORTANT NOTE: We've determined that we will ONLY
;# concentrate on the main bh and not consider the secondary as
;# it is not comparatively interesting enough to trace

;# Vanderbilt Univ.  -- VPAC40:  /home/sanchenn/IDL/highres_gasiords.pro
;# N. Nicole Sanchez -- October 18, 2015
pro highres_gasiords,mainbh,secondarybh
  
;# Parameter and Constants 
if keyword_set(mainbh) then mainbh=mainbh  else mainbh = 43553282 ;For h258
;if keyword_set(secondarybh) then secondarybh=secondarybh else secondary=435533
62
munit      = 1.84793d16                    ; M_sun
loadct,4

;# Read in fits files
file       = mrdfits('../BHAccLog.fits',1)
help,file

;# Calculate z from a factor to sort by
z          = (1. /file.afac) - 1.

;# Pull out info we want
mainbh_ind = where(file.bhiord eq mainbh)
mainbh_mgi = file.mgi[mainbh_ind]          ;Only necessary to remove badlines
mainbh_id  = file.bhiord[mainbh_ind]
mainbh_gid = file.gasiord[mainbh_ind]
mainbh_mbhf = file.mbhf[mainbh_ind] * munit ;M_sun
mainbh_dmq = file.dmacc[mainbh_ind] * munit ;M_sun
mainbh_z   = z[mainbh_ind]
;# There are some bad lines where mgi=0, remove these & redefine arrays
goodlines  = where(mainbh_mgi ne 0)
mainbh_id  = mainbh_id[goodlines]
mainbh_gid = mainbh_gid[goodlines]
mainbh_mbhf = mainbh_mbhf[goodlines]
mainbh_dmq = mainbh_dmq[goodlines]
mainbh_z   = mainbh_z[goodlines]
;# Sort in chronological order and redefine arrays
z_ind      = sort(mainbh_z)  ;returns indeces
mainbh_id  = mainbh_id[z_ind]
mainbh_gid = mainbh_gid[z_ind]
mainbh_mbhf = mainbh_mbhf[z_ind] 
mainbh_dmq = mainbh_dmq[z_ind]
mainbh_z   = mainbh_z[z_ind]
print,'Starting redshift',mainbh_z[n_elements(mainbh_z)-1]
print,'Starting mass',mainbh_mbhf[n_elements(mainbh_mbhf)-1]
print,'Ending redshift',mainbh_z[0]
print,'Ending mass',mainbh_mbhf[0]


;# If there is a secondary black hole to trace
if keyword_set(secondarybh) then begin
   secondarybh = secondarybh
   secbh_ind = where(file.bhiord eq secondarybh)
   secbh_mgi = file.mgi[secbh_ind] ;Only necessary to remove badlines
   secbh_id  = file.bhiord[secbh_ind]
   secbh_gid = file.gasiord[secbh_ind]
   secbh_mbhf = file.mbhf[secbh_ind] * munit ;M_sun
   secbh_dmq = file.dmacc[secbh_ind] * munit ;M_sun
   secbh_z   = z[secbh_ind]
   ;# There are some bad lines where mgi=0, remove these & redefine arrays
   goodlines  = where(secbh_mgi ne 0)
   secbh_id  = secbh_id[goodlines]
   secbh_gid = secbh_gid[goodlines]
   secbh_mbhf = secbh_mbhf[goodlines]
   secbh_dmq = secbh_dmq[goodlines]
   secbh_z   = secbh_z[goodlines]
   ;# Sort in chronological order and redefine arrays
   z_ind     = sort(secbh_z)
   secbh_id  = secbh_id[z_ind]
   secbh_gid = secbh_gid[z_ind]
   secbh_mbhf = secbh_mbhf[z_ind]
   secbh_dmq = secbh_dmq[z_ind]
   secbh_z   = secbh_z[z_ind]
   print,'Secondary starting redshift',secbh_z[n_elements(secbh_z)-1]
   print,'Secondary ending redshift',secbh_z[0]
   
   allbh_id  = [0]
   allbh_gid = [0]
   allbh_mbhf = [0]
   allbh_dmq = [0]
   allbh_z   = [0]
   for i=0,n_elements(mainbh_id)-1 do begin
      if (mainbh_z[i] lt secbh_z[0]) then begin
         new_id  = mainbh_id[i]
         new_gid = mainbh_gid[i]
         new_mbhf = mainbh_mbhf[i]
         new_dmq = mainbh_dmq[i]
         new_z   = mainbh_z[i]
         j=0
         ;print,new_z
      endif else begin
         new_id  = secbh_id[j]
         new_gid = secbh_gid[j]
         new_mbhf = secbh_mbhf[j]
         new_dmq = secbh_dmq[j]
         new_z   = secbh_z[j]
         ;print,new_z
         j = j+1
      endelse
      
      allbh_id  = [allbh_id,new_id]
      allbh_gid = [allbh_gid,new_gid]
      allbh_mbhf = [allbh_mbhf, new_mbhf]
      allbh_dmq = [allbh_dmq,new_dmq]
      allbh_z   = [allbh_z,new_z]
   endfor

   allbh_id  = allbh_id[1:n_elements(allbh_id)-1]
   allbh_gid = allbh_gid[1:n_elements(allbh_gid)-1]
   allbh_mbhf = allbh_mbhf[1:n_elements(allbh_mbhf)-1]
   allbh_dmq = allbh_dmq[1:n_elements(allbh_dmq)-1]
   allbh_z   = allbh_z[1:n_elements(allbh_z)-1]

endif else begin
   
   allbh_id  = mainbh_id
   allbh_gid = mainbh_gid
   allbh_mbhf = mainbh_mbhf
   allbh_dmq = mainbh_dmq
   allbh_z   = mainbh_z

endelse

;print,'Staring mass',allbh_mbhf[n_elements(allbh_mbhf)-1]
;print,'Starting redshift',mainbh_z[n_elements(mainbh_z)-1]
;print,'Ending mass',allbh_mbhf[0]
;print,'Ending Redshift',allbh_z[0]


;# The following is pulled directly from Jillian's gasiords.pro
;# for low res h258; edited for above variables 
unique_iords = allbh_id[uniq(allbh_id,sort(allbh_id))]

n=n_elements(unique_iords)
; number of bhs
print,n,' black holes'
close,1

for i=0,n-1 do begin
    thisbh = where(allbh_id eq unique_iords[i])
    eatengas = allbh_gid[thisbh]
    eatenmass = allbh_dmq[thisbh]
    ; only unique iords here.
    uniquegas=uniq(eatengas,sort(eatengas)) ; indices
    eacheatengas=eatengas[uniquegas]
    filename='gasiords.'+trim(unique_iords[i])
    openw,1,filename
    for j=0L,n_elements(eacheatengas)-1 do begin
        eacheatenmass=eatenmass[where(eatengas eq eacheatengas[j])]
        printf,1,eacheatengas[j],total(eacheatenmass)
    endfor
    close,1
endfor


spawn,'ls gasiords.* > gasiords.list'
;# End of Jillian's code


;# Another attempt to just pull out gasiords and mass;
;# Use above method, but figure out why that works better
;filename='gasiords.good.all_fromNicolesway'
;openw,1,filename
;for i=0,n_elements(allbh_gid)-1 do begin
;   printf,1,allbh_gid[i],allbh_dmq[i]
;endfor
;close,1


;# Create a pseudo out.bh with bh ids, gas ids, time, dmq
filename='out.bh'
;openw,1,filename
;for i=0,n_elements(allbh_z)-1 do begin
;   printf,1,'BHiord',allbh_id[i],'gasiord',allbh_gid[i],'eaten mass/dmq',allbh
_dmq[i],'redshift',al;lbh_z[i]
;endfor
;close,1


;# Make .fits files! Too big otherwise
mwrfits, allbh_id, 'centralbh_bhiords.fits', /create
mwrfits, allbh_gid, 'centralbh_gasiords.fits', /create
mwrfits, allbh_mbhf, 'centralbh_bhmass.fits', /create
mwrfits, allbh_dmq, 'centralbh_eatenmass.fits', /create
mwrfits, allbh_z, 'centralbh_redshift.fits', /create




end
\end{verbatim}

\subsection{highres\_massgastime.pro}
\begin{verbatim}
;# This script is a redux of Jillian's massgastime.pro (find
;# in Vanderbilt VPAC40 /home/sanchenn/IDL/massgastime.pro
;# which works for low res, Gasoline simulation outputs)
;# This script works for high res, CHANGA files instead.

;# What we need:
;#      BHIORDS           (ids of bh)
;#      EATTIME/TIME_Z    (accretion time)
;#      GASIORDS          (ids of gas particles) 
;#      DM/DMQ/EATENMASS  (mass accreted from gas particles)

;# Our steps:
;# 1. Read in .fits for all above info
;#    NOTE: This is chronological order

;# Vanderbilt Univ.  -- VPAC40:
;#                      /home/sanchenn/IDL/highres_massgastime.pro
;# N. Nicole Sanchez -- Created Oct. 2015; Last Edit April 5, 2016
pro highres_massgastime,plotter=plotter
;loadct,4
loadct,39  ;#Jillian's Colors

;# Constants
tunit=1.223128407d18/3.1556926d7/1d9
;tunit = 1.223128407d18/3.1556926d7/1d9  ; Gyr 
munit = 1.84793d16  ; M_sun

;# Read above listed quantities
bhiords   = mrdfits('centralbh_bhiords.fits',0)
gasiords  = mrdfits('centralbh_gasiords.fits',0)  
bhmass    = mrdfits('centralbh_bhmass.fits',0)    ; M_sun
eatenmass = mrdfits('centralbh_eatenmass.fits',0) ; M_sun
time_z    = mrdfits('centralbh_redshift.fits',0)  ; In chronological order
time_afac = (1/(time_z + 1))
time_gyr  = mrdfits('centralbh_timeGyr.fits',0)   ; Gyr

;# Calculate time in Gyr
; times and redshift, for the z axis on the plots
;readcol,'../times.list',gyr,z,/silent
zplotnames=['6','4','2','1','0.5','0']
zplotnum=[6,4,2,1,0.5,0]
tickvalues=interpol(time_gyr,time_z,zplotnum) 
tickvalues[n_elements(tickvalues)-1]=time_gyr[n_elements(time_gyr)-1]
;print,time_gyr[0]
;stop

;# Read in times to create the x-axis for our plots later
;# Following is from Jillian's massgastime.pro
zplotnames=['6','4','2','1','0.5','0']
zplotnum=[6,4,2,1,0.5,0]
tickvalues=interpol(time_gyr,time_z,zplotnum)
tickvalues[n_elements(tickvalues)-1]=time_gyr[n_elements(gyr)-1]

;# Read in my .fits files (created by highres_gasiords.pro)
restore,'allgas.sav' ; Gas is in units of M_sun
;help,allgas
zs = allgas.zaccr
ngas = n_elements(allgas.iords)

file       = mrdfits('../BHAccLog.fits',1)
;help,file

uniq_ind = uniq(file.bhiord)
;print,'Number of unique BH indeces',n_elements(uniq_ind)
uniq_ids = file.bhiord[UNIQ(file.bhiord, SORT(file.bhiord))]
print,'Number of unique BH IDs',n_elements(uniq_ids)

;print,min(file.bhiord)
;print,max(file.bhiord)
gasuniq_ids = uniq(file.gasiord)
uniq_gas = file.gasiord[UNIQ(file.gasiord, SORT(file.gasiord))]
;print,'Number of unique gas iords,',n_elements(uniq_gas),' should not equal al
l of them',n_elements(file.gasiord),' since file.gasiord is across all time'

uniq_gasiords = gasiords[UNIQ(gasiords, SORT(gasiords))]
print,'Number of unique gas IDs in gasiords',uniq_gasiords[0:100]


match_multi,allgas.iords,gasiords,all1
print,'Number of gas IDS in gasiords',n_elements(gasiords)
; We want to use gasiords because it directly compares to eatenmass
; and redshift and we know the BH can accrete from a gas particle
; multiple times by taking "bites" out of it
print,'Number of unique gas IDs in allgas.sav',ngas
print,'Number of unique gas IDs in BHAccLog',n_elements(uniq_gas)
; This number SHOULD be larger than the uniq gas Ids value for both
; allgas.sav and gasiords since they both only consider halo 1 and
; BHAccLog includes ALL BHs
print,'Number of matched gas IDs between allgas.sav & BHAccLog.fits',n_elements
(all1)
print,'Total BH accreted particles',n_elements(gasiords)
print,'BH accreted categorized  particles',n_elements(allgas.iords)
;stop

;# Divide Gas into Smooth and Clumpy
;# CLUMPY:
match_multi,allgas.clumpy,gasiords,c11
clumpybhtimes = time_z[c11]
clumpybhyears = time_Gyr[c11]
clumpybhmass  = eatenmass[c11]
;# SHOCKED
match_multi,allgas.shocked,gasiords,sh11
shockedbhtimes = time_z[sh11]
shockedbhyears = time_Gyr[sh11]
shockedbhmass  = eatenmass[sh11]
;# COLD
match_multi,allgas.cold,gasiords,cold11
coldbhtimes = time_z[cold11]
coldbhyears = time_Gyr[cold11]
coldbhmass  = eatenmass[cold11]
;# EARLY
match_multi,allgas.early,gasiords,early1
earlybhtimes = time_z[early1]
earlybhyears = time_Gyr[early1]
earlybhmass = eatenmass[early1]
print,'How many early gas in allgas?',n_elements(allgas.early)
print,'How many shocked gas in allgas?',n_elements(allgas.shocked)
print,'How many clumpy gas in allgas?',n_elements(allgas.clumpy)
print,'How many cold gas in allgas?',n_elements(allgas.cold)
stop

;# Put back in chronological order
;# CLUMPY
sortclumpy    = sort(clumpybhtimes)
clumpybhtimes = clumpybhtimes[sortclumpy]
clumpybhyears = clumpybhyears[sortclumpy]
clumpybhmass  = clumpybhmass[sortclumpy]
clumpybhtimes = reverse(clumpybhtimes)
clumpybhyears = reverse(clumpybhyears)
clumpybhmass  = reverse(clumpybhmass)
;# SHOCKED
sortshocked    = sort(shockedbhtimes)
shockedbhtimes = shockedbhtimes[sortshocked]
shockedbhyears = shockedbhyears[sortshocked]
shockedbhmass  = shockedbhmass[sortshocked]
shockedbhtimes = reverse(shockedbhtimes)
shockedbhyears = reverse(shockedbhyears)
shockedbhmass  = reverse(shockedbhmass)
;# COLD
sortcold    = sort(coldbhtimes)
coldbhtimes = coldbhtimes[sortcold]
coldbhyears = coldbhyears[sortcold]
coldbhmass  = coldbhmass[sortcold]
coldbhtimes = reverse(coldbhtimes)
coldbhyears = reverse(coldbhyears)
coldbhmass  = reverse(coldbhmass)
;# EARLY
sortearly    = sort(earlybhtimes)
earlybhtimes = earlybhtimes[sortearly]
earlybhyears = earlybhyears[sortearly]
earlybhmass  = earlybhmass[sortearly]
earlybhtimes = reverse(earlybhtimes)
earlybhyears = reverse(earlybhyears)
earlybhmass  = reverse(earlybhmass)

;# TOTAL
eatenmass_rev = reverse(eatenmass)
time_z_rev    = reverse(time_z)
time_Gyr_rev    = reverse(time_Gyr)
;# BH TOTAL
;bhmass_rev    = reverse(bhmass)
bhmass_rev    = bhmass

;# Add and get cumulative totals
coldtotal    = total(coldbhmass,/cumulative)
shockedtotal = total(shockedbhmass,/cumulative)
clumpytotal  = total(clumpybhmass,/cumulative)
earlytotal   = total(earlybhmass,/cumulative)
alltotal     = total(eatenmass_rev,/cumulative)
bhtotal      = total(bhmass_rev,/cumulative)


;print,clumpybhtimes[0:50]
;print,shockedbhtimes[0:50]
;print,coldbhtimes[0:50]
;print,earlybhtimes
;print,time_z_rev[0:50]

eatenmasstotal = total(eatenmass,/cumulative)
print,'clumpy + cold + shocked + early ids',n_elements(clumpytotal),n_elements(
coldtotal),n_elements(shockedtotal),n_elements(earlytotal),' equals ',n_element
s(clumpytotal) + n_elements(coldtotal) + n_elements(shockedtotal) + n_elements(
earlytotal),' should equal all gas ids',n_elements(gasiords)
print,'clumpy + cold + shocked + early times',n_elements(clumpybhtimes),n_eleme
nts(coldbhtimes),n_elements(shockedbhtimes),n_elements(earlybhtimes),' equals '
,n_elements(clumpybhtimes) + n_elements(coldbhtimes) + n_elements(shockedbhtime
s) + n_elements(earlybhtimes),' should equal all gas ids',n_elements(time_z_rev
)
;stop


if keyword_set(plotter) then paperplot,filename='cumumassvstime.ps'
plot,time_Gyr,reverse(alltotal),xtit='Time [Gyr]',ytit='Cumulative BH Mass (M'+
sunsymbol()+')',yra=[1000,1e9],/ylog,xra=[0,time_gyr[0]],xstyle=9;,color=0
oplot,time_Gyr,bhmass,linestyle=2
oplot,coldbhyears,coldtotal,color=90,linestyle=3
;oplot,earlybhtimes,earlytotal,linestyle=2,color=255
oplot,clumpybhyears,clumpytotal,color=140
oplot,shockedbhyears,shockedtotal,color=240,linestyle=1
legend,['Total BH Mass','Total Gas','Cold','Clumpy','Shocked'],lines=[2,0,0,3,0
],color=[0,0,80,140,240],charsize=1,/top,/left
axis,xaxis=1,xtit='Redshift',xtickv=tickvalues,xticks=n_elements(zplotnames)-1,
xtickname=zplotnames
if keyword_set(plotter) then paperplot,/close
stop

;REDSHIFT vs mass
if keyword_set(plotter) then paperplot,filename='cumumassvsz.ps'
plot,time_z,reverse(alltotal),xtit='Redshift',ytit='Cumulative BH Mass (M'+suns
ymbol()+')',yra=[1000,1e8],/ylog,xra=[14,0];,color=0
oplot,time_z,bhmass,linestyle=2
oplot,coldbhtimes,coldtotal,color=80,linestyle=3
oplot,clumpybhtimes,clumpytotal,color=140
oplot,shockedbhtimes,shockedtotal,color=240,linestyle=1
legend,['Total BH Mass','Total Gas','Cold','Clumpy','Shocked'],lines=[2,0,0,3,0
],color=[0,0,80,140,240],charsize=1,/top,/left
if keyword_set(plotter) then paperplot,/close

axis,xaxis=1,xtit='Redshift',xtickv=tickvalues,xticks=n_elements(zplotnames)-1,
xtickname=z
stop


;# Create mass vs time plots: need to bin data
;# BIN TIME
nbins = 100
maxtime = max(time_z)
mintime = min(time_z)
bindivision = (maxtime-mintime)/nbins
tinds = indgen(nbins)*bindivision+mintime  ; indices for time array.
bindmtime = tinds
;# BIN MASS, use sum of mass in each bin set
bindm = fltarr(nbins)
bincoldbhmass =  fltarr(nbins)
binclumpybhmass =  fltarr(nbins)
binshockedbhmass =  fltarr(nbins)
for i=0,nbins-2 do begin
    inds = where(time_z ge (i*bindivision+mintime) AND time_z lt ((i+1)*bindivi
sion)+mintime,ninds)
    if ninds ne 0 then bindm[i] = total(eatenmass[inds])
    coldinds = where(coldbhtimes ge (i*bindivision+mintime) AND coldbhtimes lt 
((i+1)*bindivision)+mintime,ncoldinds)
    if ncoldinds ne 0 then bincoldbhmass[i] = total(coldbhmass[coldinds])
    clumpyinds = where(clumpybhtimes ge (i*bindivision+mintime) AND clumpybhtim
es lt ((i+1)*bindivision)+mintime,nclumpyinds)
    if nclumpyinds ne 0 then binclumpybhmass[i] = total(clumpybhmass[clumpyinds
])
    shockedinds = where(shockedbhtimes ge (i*bindivision+mintime) AND shockedbh
times lt ((i+1)*bindivision)+mintime,nshockedinds)
    if nshockedinds ne 0 then binshockedbhmass[i] = total(shockedbhmass[shocked
inds])
endfor

;# Mass vs Redshift
if keyword_set(plotter) then paperplot,filename='massvstime.ps'
;print,bindmtime
plot,bindmtime,bindm,xtit='Redshift',ytit='Mass Accreted by Central BH',/ylog,y
ra=[1,1e6],xra=[20,0]
oplot,bindmtime,binclumpybhmass,color=140;,psym=3
oplot,bindmtime,bincoldbhmass,color=90   ;,psym=3
oplot,bindmtime,binshockedbhmass,color=240 ;,psym=3
legend,['all','cold','shocked','clumpy'],lines=[0,0,0,0],color=[0,90,240,140],c
harsize=1,/right,/bottom
if keyword_set(plotter) then paperplot,/close
stop

;# Cumulative Mass Plot
;# Need to convert all times to a factor to multiply by tunit
if keyword_set(plotter) then paperplot,filename='allmassgas.ps'
plot,time_z,reverse(alltotal),xtit='Redshift',ytit='Cumulative BH Mass (M'+suns
ymbol()+')',/ylog,yra=[100,1e8],xra=[20,0];,xstyle=8  ;puts on upper x-axis
oplot,time_z,clumpytotal,color=125
oplot,time_z,bhmass,lines=2,color=0
oplot,time_z,coldtotal,lines=0,color=80
oplot,time_z,shockedtotal,lines=0,color=240
oplot,time_z,earlytotal,color=5

legend,['total BH mass','total gas','cold','shocked','clumpy'],lines=[2,0,0,0,0
],color=[0,0,80,240,125],charsize=1,/top,/left
;# Fix following line for upper x-axis
;axis,xaxis=1,xtit='Redshift',xtickv=tickvalues,xticks=n_elements(zplotnames)-1
,xtickname=zplotnames
if keyword_set(plotter) then paperplot,/close
stop

; ### test that the parts add up to the whole. ###

nclumpy = n_elements(clumpytotal)
ncold = n_elements(coldtotal)
nshocked = n_elements(shockedtotal)
ndm = n_elements(alltotal)
nearly = n_elements(earlytotal)

print,'Does the final mass of alltotal',alltotal[ndm-1],' equal the final masse
s of all the others?',clumpytotal[nclumpy-1]+coldtotal[ncold-1]+shockedtotal[ns
hocked-1]+earlytotal[nearly-1]

print,'gas mass fractions '
print,'cold/total = ',coldtotal[ncold-1]/alltotal[ndm-1]
print,'clumpy/total = ',clumpytotal[nclumpy-1]/alltotal[ndm-1]
print,'shocked/total = ',shockedtotal[nshocked-1]/alltotal[ndm-1]
print,'early/total = ',earlytotal[nearly-1]/alltotal[ndm-1]

print,'gas mass fractions (weird?)'
print,'cold/total = ',total(coldtotal)/total(alltotal)
print,'clumpy/total = ',total(clumpytotal)/total(alltotal)
print,'shocked/total = ',total(shockedtotal)/total(alltotal)
print,'early/total = ',total(earlytotal)/total(alltotal)

print,'does the total mass of alltotal',total(alltotal),' equal the added sums 
of all the others?','cold',total(coldtotal),' clumpy',total(clumpytotal),' shoc
ked',total(shockedtotal),' early',total(earlytotal)
print,'total =',total(coldtotal)+total(clumpytotal)+total(shockedtotal)+total(e
arlytotal)


print,'gas number fractions '
print,'cold/total = ',n_elements(coldbhmass)
;/n_elements(alltotal)
print,'clumpy/total = ',n_elements(clumpybhmass)
;/n_elements(alltotal)
print,'shocked/total = ',n_elements(shockedbhmass)
;/n_elements(alltotal)
print,'early/total = ',n_elements(earlybhmass)/n_elements(alltotal)

end
\end{verbatim}

\section{Python Scripts}
Unlike the IDL scripts upon which I had a basic structure to expand upon, the angular momentum and infall time analyses required a larger departure from previous studies and was better suited to a Python based analysis. Note that each of the following scripts had a slightly varied version for each of the different galaxies described above. See https://github.com/nnicolesanchez/masterspythonscripts for all iterations and variations.

\subsection{angmom.py}
\begin{verbatim}
# This script needs to create an easily searchable structure that matches
#      - Gas Particle IDs
#      - Gas Particle Origin (Cold, Clupmy, Shocked, Early)
#      - Accretion time of Gas Particle
#      - Angular momentum in x, y, and z.
#      - Calculated Total Angular momentum

# Plan:
# Read in files for each timestep.
# Determine which gas particles end up in Main Halo 1 
# Determine which gas particles end up in Main SMBH
# Match accretion times to each particle
# Match x, y, z, angular momentum components to each particle
# Calculate total angular momentum for each particle
# Print out into an easily searchable structure


# Vanderbilt Univ.  -- VPAC40:  /astro1/nicole/lowresh258/angmom.py 
# N. Nicole Sanchez -- January 20, 2016 
import matplotlib.pyplot as plt
import pyfits  as pyf
import numpy   as np
import pynbody as pyn

haloid_file = 'h258.cosmo50cmb.1536gst1MbwKBH.haloid_reverse.dat'
ts_numbers  = np.loadtxt('../timestep_numbers.dat',unpack=True,dtype=str)

# Read in gas particle data files:  
h1_iords_all      = pyf.getdata('../grp1.allgas.iord.fits')    # all gas IDs en
ding in MH
h1_iords_clumpy   = pyf.getdata('../clumpy.accr.iord.fits')    # all clumpy gas
 IDs
h1_iords_smooth   = pyf.getdata('../smooth.accr.iord.fits')
h1_iords_shock    = pyf.getdata('../shocked.iord.fits')
h1_iords_unshock  = pyf.getdata('../unshock.iord.fits')
h1_iords_early    = pyf.getdata('../early.iord.fits')          # all early gas 
IDs (in MH)
h1_accrz          = pyf.getdata('../grp1.accrz.fits')

h1_early_mask     = np.in1d(h1_iords_all,h1_iords_early)
h1_notearly_mask  = ~h1_early_mask
h1_iords_notearly = h1_iords_all[h1_notearly_mask]
#print np.argsort(h1_iords_all)
#quit()

bh_iords_all      = np.loadtxt('../centralbh/gasiords.good.all')

#>>> np.unique([1, 1, 2, 2, 3, 3])
#array([1, 2, 3])
print len(h1_accrz)
accruniq = np.unique(h1_accrz)
print len(accruniq)                 # Since there are 44, timesteps = accrtime

print len(h1_iords_all)
halouniq = np.unique(h1_iords_all)
print len(halouniq)                 # So allhalogas are unique!

list_of_ts = np.loadtxt('../'+haloid_file,dtype='str',unpack=True)
timesteps  = list_of_ts[0]
haloes     = list_of_ts[1] 

# INITIALIZE OUTPUT ARRAYS
cumu_iords  = []
cumu_angmom = []
cumu_accrz  = []
cumu_bh_iords  = []
cumu_bh_angmom = [] 

num_new_gas_in_ts = []
redshifts = []

# Note: Starting at 1 b/c there were ONLY EARLY particles in i=0 
for i in range(1,len(timesteps)):
    timestep = '../' + str(timesteps[i])
    halo     = haloes[i]    

# Pynbody load snapshot of timestep and halo
    snapshot = pyn.load(timestep)
    snapshot.properties
    snapshot.loadable_keys()
    snapshot.physical_units()
    
    allhalos = snapshot.halos()            
    halo     = int(halo)
    print timestep
    print int(halo)
    h1       = allhalos[halo]
    pyn.analysis.halo.center(h1,mode='ssc')
    pyn.analysis.angmom.faceon(h1)
    print('A factor =',h1.properties['a'])
    afactor  = h1.properties['a']
    redshift = (1./afactor)-1
    print('Time =',h1.properties['time'].in_units('Gyr'),'Gyr')
#print(snapshot.loadable_keys())         # To check units
#print(snapshot.derivable_keys()[:20])

    ts_iords = h1.g['iord']
    ts_v     = h1.g['vel']
    ts_xyz     = h1.g['pos']
#print ts_xyz

    print 'IDs in the timestep',len(ts_iords)
    print 'IDs in the halo',len(h1_iords_notearly)
    ts_iords_mask = np.in1d(h1_iords_notearly,ts_iords)
    print 'mask values',len(ts_iords_mask)

    ts_uniq = np.unique(ts_iords)
    print 'unique in timestep',len(ts_uniq) 
    
    print 'times in the halo',len(h1_accrz)
    ts_iords_in_h1    = h1_iords_notearly[ts_iords_mask]    
    ts_accrz_notearly = h1_accrz[ts_iords_mask]
    print 'IDs DEFINITELY in the timestep and main halo',len(ts_iords_in_h1)
    print 'accr times of IDs in timestep',len(ts_accrz_notearly)

    ts_early_mask     = np.in1d(ts_iords,h1_iords_early)
    print 'early IDs in timestep',sum(ts_early_mask)
    ts_notearly_mask  = ~ts_early_mask


    ts_iords_notearly = ts_iords[ts_notearly_mask]
    ts_v_notearly     = ts_v[ts_notearly_mask]
    ts_xyz_notearly   = ts_xyz[ts_notearly_mask]
#print ts_xyz_notearly

    ts_angmom = np.cross(ts_xyz_notearly, ts_v_notearly)
    print 'angmom of IDs in timestep',len(ts_angmom)

#print ts_iords_notearly
#print ts_iords_in_h1
#print ts_angmom

# If you want to time sort:
#ts_accr_sort_ind  = np.argsort(ts_accrz_notearly)
#ts_accrz_sorted   = ts_accrz_notearly[ts_accr_sort_ind]
#ts_iords_zsorted  = ts_iords_notearly[ts_accr_sort_ind]
#ts_angmom_zsorted = ts_angmom[ts_accr_sort_ind]

    ts_angmom_total   = []
    for j in range(0,len(ts_angmom)):
        ts_angmom_total.append(((ts_angmom[j][0])**2. + (ts_angmom[j][1])**2. +
 (ts_angmom[j][2])**2.)**(1./2.))
    ts_angmom_total   =  np.array(ts_angmom_total)

    ts_angmom_amsort_ind = np.argsort(ts_angmom_total)
    ts_angmom_amsorted   = ts_angmom_total[ts_angmom_amsort_ind]
    ts_accrz_amsorted    = ts_accrz_notearly[ts_angmom_amsort_ind]
    ts_iords_amsorted    = ts_iords_notearly[ts_angmom_amsort_ind]

    ts_am_cumu = np.cumsum(ts_angmom_amsorted)
    ts_am_cumu_norm = ts_am_cumu / ts_am_cumu[len(ts_am_cumu)-1]
    print('this should be 1:',ts_am_cumu_norm[len(ts_am_cumu)-1])


####################################################################
####################################################################
# ALL GAS IN TIMESTEP GALAXY GAS --- GALAXY GAS                    #
####################################################################
    unshock_mask = np.in1d(ts_iords_amsorted,h1_iords_unshock)
    clumpy_mask  = np.in1d(ts_iords_amsorted,h1_iords_clumpy)
    shock_mask   = np.in1d(ts_iords_amsorted,h1_iords_shock)

# UNSHOCKED
    ts_unshock_angmom = ts_angmom_amsorted[unshock_mask]
    ts_unshock_am_cumu = np.cumsum(ts_unshock_angmom)
    ts_unshock_am_cumu_norm = ts_unshock_am_cumu / ts_unshock_am_cumu[len(ts_un
shock_am_cumu)-1]
# CLUMPY
    ts_clumpy_angmom  = ts_angmom_amsorted[clumpy_mask]
    ts_clumpy_am_cumu = np.cumsum(ts_clumpy_angmom)
    ts_clumpy_am_cumu_norm = ts_clumpy_am_cumu / ts_clumpy_am_cumu[len(ts_clump
y_am_cumu)-1]
# SHOCKED
    ts_shock_angmom   = ts_angmom_amsorted[shock_mask]
    ts_shock_am_cumu = np.cumsum(ts_shock_angmom)
    ts_shock_am_cumu_norm = ts_shock_am_cumu / ts_shock_am_cumu[len(ts_shock_am
_cumu)-1]

    print('this is the number of gas particles in the main halo at this ts',len
(ts_unshock_angmom)+len(ts_clumpy_angmom)+len(ts_shock_angmom),'which should eq
ual',len(ts_iords_in_h1))

# GALAXY GAS PLOTTING
#plt.hist(ts_am_cumu_norm)
#plt.plot(ts_angmom_amsorted,ts_am_cumu_norm,'--',color='Black') 
    plt.plot(ts_unshock_angmom,ts_unshock_am_cumu_norm,'-',color='Blue')
    plt.plot(ts_clumpy_angmom,ts_clumpy_am_cumu_norm,'-',color='Green')
    plt.plot(ts_shock_angmom,ts_shock_am_cumu_norm,'-',color='Red')
    plt.ylabel('Normalized Cumulative Distribution')
    plt.xlabel('Angular Momentum |j|')
    plt.xlim([0,25000])
#plt.show()

####################################################################
# ALL GAS IN TIMESTEP GALAXY GAS --- BLACK HOLE GAS                #
####################################################################
    bh_iords_mask   = np.in1d(ts_iords_amsorted,bh_iords_all)
    bh_iords_ts     = ts_iords_amsorted[bh_iords_mask]
    bh_angmom_ts    = ts_angmom_amsorted[bh_iords_mask]
    bh_unshock_mask = np.in1d(bh_iords_ts,h1_iords_unshock)
    bh_clumpy_mask  = np.in1d(bh_iords_ts,h1_iords_clumpy)
    bh_shock_mask   = np.in1d(bh_iords_ts,h1_iords_shock)
    
# UNSHOCKED
    bh_unshock_angmom  = bh_angmom_ts[bh_unshock_mask]
    bh_unshock_am_cumu = np.cumsum(bh_unshock_angmom)
    bh_unshock_am_cumu_norm = bh_unshock_am_cumu / bh_unshock_am_cumu[len(bh_un
shock_am_cumu)-1]
# CLUMPY
    bh_clumpy_angmom  = bh_angmom_ts[bh_clumpy_mask]
    bh_clumpy_am_cumu = np.cumsum(bh_clumpy_angmom)
    bh_clumpy_am_cumu_norm = bh_clumpy_am_cumu / bh_clumpy_am_cumu[len(bh_clump
y_am_cumu)-1]
# SHOCKED
    bh_shock_angmom  = bh_angmom_ts[bh_shock_mask]
    bh_shock_am_cumu = np.cumsum(bh_shock_angmom)
    bh_shock_am_cumu_norm = bh_shock_am_cumu / bh_shock_am_cumu[len(bh_shock_am
_cumu)-1]

    print('this is the number of gas particles in the black hole at this ts',le
n(bh_unshock_angmom)+len(bh_clumpy_angmom)+len(bh_shock_angmom))

# BLACK HOLE GAS PLOTTING
    plt.plot(bh_unshock_angmom,bh_unshock_am_cumu_norm,'--',color='Blue')
    plt.plot(bh_clumpy_angmom,bh_clumpy_am_cumu_norm,'--',color='Green')
    plt.plot(bh_shock_angmom,bh_shock_am_cumu_norm,'--',color='Red')
    plt.ylabel('Normalized Cumulative Distribution')
    plt.xlabel('Angular Momentum |j|')
    plt.title('z = '+str(redshift))
    plt.xlim([0,25000])
    plt.savefig('angmom_cumudist_tsall_'+ts_numbers[i]+'.ps')
    plt.clf()
#    plt.show()


####################################################################
####################################################################
# ACCUMULATED GAS ACROSS TIMESTEPS --- GALAXY GAS                  #
####################################################################
    cumu_iords_mask    = np.in1d(ts_iords_amsorted,cumu_iords)
    cumu_newiords_mask = ~cumu_iords_mask
    new_iords_in_ts    = ts_iords_amsorted[cumu_newiords_mask]
    new_angmom_in_ts   = ts_angmom_amsorted[cumu_newiords_mask]
    print('accumulated # of iords',len(cumu_iords),'plus timestep iords',len(ne
w_iords_in_ts))

    if len(cumu_iords) == 0 :
        cumu_iords  = new_iords_in_ts
        cumu_angmom = new_angmom_in_ts
    else :    
        cumu_iords  = np.array(cumu_iords)
        cumu_angmom = np.array(cumu_angmom)
        new_iords_in_ts_all  = np.array(new_iords_in_ts)
        new_angmom_in_ts_all = np.array(new_angmom_in_ts)
        cumu_iords  = np.concatenate((cumu_iords,new_iords_in_ts_all), axis=0)
        cumu_angmom = np.concatenate((cumu_angmom,new_angmom_in_ts_all), axis=0
)
    print('should equal',len(cumu_iords))

# Reorder them in angmom order
    cumu_amsort_ind    = np.argsort(cumu_angmom)
    cumu_angmom_amsort = cumu_angmom[cumu_amsort_ind]
    cumu_iords_amsort  = cumu_iords[cumu_amsort_ind]
# START PLOTTING
    cumu_unshock_mask = np.in1d(cumu_iords_amsort,h1_iords_unshock)
    cumu_clumpy_mask  = np.in1d(cumu_iords_amsort,h1_iords_clumpy)
    cumu_shock_mask   = np.in1d(cumu_iords_amsort,h1_iords_shock)
# UNSHOCKED
    cumu_unshock_angmom  = cumu_angmom_amsort[cumu_unshock_mask]
    cumu_unshock_am_dist = np.cumsum(cumu_unshock_angmom)
    cumu_unshock_am_dist_norm = cumu_unshock_am_dist / cumu_unshock_am_dist[len
(cumu_unshock_am_dist)-1]
# CLUMPY
    cumu_clumpy_angmom  = cumu_angmom_amsort[cumu_clumpy_mask]
    cumu_clumpy_am_dist = np.cumsum(cumu_clumpy_angmom)
    cumu_clumpy_am_dist_norm = cumu_clumpy_am_dist / cumu_clumpy_am_dist[len(cu
mu_clumpy_am_dist)-1]
# SHOCKED
    cumu_shock_angmom  = cumu_angmom_amsort[cumu_shock_mask]
    cumu_shock_am_dist = np.cumsum(cumu_shock_angmom)
    cumu_shock_am_dist_norm = cumu_shock_am_dist / cumu_shock_am_dist[len(cumu_
shock_am_dist)-1]

# GALAXY GAS PLOTTING
    plt.plot(cumu_unshock_angmom,cumu_unshock_am_dist_norm,'-',color='Blue')
    plt.plot(cumu_clumpy_angmom,cumu_clumpy_am_dist_norm,'-',color='Green')
    plt.plot(cumu_shock_angmom,cumu_shock_am_dist_norm,'-',color='Red')
    plt.ylabel('Normalized Cumulative Distribution')
    plt.xlabel('Angular Momentum |j|')
    plt.xlim([0,25000])

####################################################################
# ACCUMULATED GAS ACROSS TIMESTEPS --- BLACK HOLE GAS              #
####################################################################
    cumu_bh_iords_mask    = np.in1d(bh_iords_ts,cumu_bh_iords)
    cumu_bh_newiords_mask = ~cumu_bh_iords_mask
    new_bh_iords_in_ts    = bh_iords_ts[cumu_bh_newiords_mask]
    new_bh_angmom_in_ts   = bh_angmom_ts[cumu_bh_newiords_mask]
    print('accumulated # of iords',len(cumu_bh_iords),'plus timestep iords',len
(new_bh_iords_in_ts))
    if len(cumu_iords) == 0 :
        cumu_bh_iords  = new_bh_iords_in_ts
        cumu_bh_angmom = new_bh_angmom_in_ts
    else :
        cumu_bh_iords  = np.array(cumu_bh_iords)
        cumu_bh_angmom = np.array(cumu_bh_angmom)
        new_bh_iords_in_ts_all  = np.array(new_bh_iords_in_ts)
        new_bh_angmom_in_ts_all = np.array(new_bh_angmom_in_ts)
        cumu_bh_iords  = np.concatenate((cumu_bh_iords,new_bh_iords_in_ts_all),
 axis=0)
        cumu_bh_angmom = np.concatenate((cumu_bh_angmom,new_bh_angmom_in_ts_all
), axis=0)
    print('should equal',len(cumu_bh_iords))

# Reorder them in angmom order    
    cumu_bh_amsort_ind    = np.argsort(cumu_bh_angmom)
    cumu_bh_angmom_amsort = cumu_bh_angmom[cumu_bh_amsort_ind]
    cumu_bh_iords_amsort  = cumu_bh_iords[cumu_bh_amsort_ind] 

# START PLOTTING BH GAS
    cumu_bh_iords_mask   = np.in1d(cumu_bh_iords_amsort,bh_iords_all)
    cumu_bh_iords_ts     = cumu_bh_iords_amsort[cumu_bh_iords_mask]
    cumu_bh_angmom_ts    = cumu_bh_angmom_amsort[cumu_bh_iords_mask]    
    cumu_bh_unshock_mask = np.in1d(cumu_bh_iords_amsort,h1_iords_unshock)
    cumu_bh_clumpy_mask  = np.in1d(cumu_bh_iords_amsort,h1_iords_clumpy)
    cumu_bh_shock_mask   = np.in1d(cumu_bh_iords_amsort,h1_iords_shock)

# UNSHOCKED
    cumu_bh_unshock_angmom  = cumu_bh_angmom_amsort[cumu_bh_unshock_mask]
    cumu_bh_unshock_am_dist = np.cumsum(cumu_bh_unshock_angmom)
    cumu_bh_unshock_am_dist_norm = cumu_bh_unshock_am_dist / cumu_bh_unshock_am
_dist[len(cumu_bh_unshock_am_dist)-1]
# CLUMPY
    cumu_bh_clumpy_angmom  = cumu_bh_angmom_amsort[cumu_bh_clumpy_mask]
    cumu_bh_clumpy_am_dist = np.cumsum(cumu_bh_clumpy_angmom)
    cumu_bh_clumpy_am_dist_norm = cumu_bh_clumpy_am_dist / cumu_bh_clumpy_am_di
st[len(cumu_bh_clumpy_am_dist)-1]
# SHOCKED
    cumu_bh_shock_angmom  = cumu_bh_angmom_amsort[cumu_bh_shock_mask]
    cumu_bh_shock_am_dist = np.cumsum(cumu_bh_shock_angmom)
    cumu_bh_shock_am_dist_norm = cumu_bh_shock_am_dist / cumu_bh_shock_am_dist[
len(cumu_bh_shock_am_dist)-1]

# BLACK HOLE GAS PLOTTING
    plt.plot(cumu_bh_unshock_angmom,cumu_bh_unshock_am_dist_norm,'--',color='Bl
ue')
    plt.plot(cumu_bh_clumpy_angmom,cumu_bh_clumpy_am_dist_norm,'--',color='Gree
n')
    plt.plot(cumu_bh_shock_angmom,cumu_bh_shock_am_dist_norm,'--',color='Red')
    plt.ylabel('Normalized Cumulative Distribution')
    plt.xlabel('Angular Momentum |j|')
    plt.title('z = '+str(redshift))
    plt.xlim([0,25000])
    plt.savefig('angmom_cumudist_cumu_'+ts_numbers[i]+'.ps')
    plt.clf()
#    plt.show()

    print('there are',len(cumu_bh_iords),'bh iords and',len(np.unique(cumu_bh_i
ords)),'of them are unique')
    print('there are',len(cumu_iords),'gxy iords and',len(np.unique(cumu_iords)
),'of them are unique')


####################################################################
####################################################################
# NEW GAS IN TIMESTEP GALAXY GAS --- GALAXY GAS                    #
####################################################################
# Reorder them in angmom order
#    if len(cumu_iords) == 0 :
#        new_bh_iords  = new_bh_iords_in_ts
#        new_bh_angmom = new_bh_angmom_in_ts
    
    new_amsort_ind    = np.argsort(new_angmom_in_ts)
    new_angmom_amsort = new_angmom_in_ts[new_amsort_ind]
    new_iords_amsort  = new_iords_in_ts[new_amsort_ind]

# START PLOTTING
    new_unshock_mask = np.in1d(new_iords_amsort,h1_iords_unshock)
    new_clumpy_mask  = np.in1d(new_iords_amsort,h1_iords_clumpy)
    new_shock_mask   = np.in1d(new_iords_amsort,h1_iords_shock)
# UNSHOCKED
    new_unshock_angmom  = new_angmom_amsort[new_unshock_mask]
    new_unshock_am_dist = np.cumsum(new_unshock_angmom)
    if (len(new_unshock_am_dist) >= 2):
        print(new_unshock_am_dist)
        print(len(new_unshock_am_dist))
        new_unshock_am_dist_norm = new_unshock_am_dist / new_unshock_am_dist[le
n(new_unshock_am_dist)-1]
        plt.plot(new_unshock_angmom,new_unshock_am_dist_norm,'-',color='Blue')
# CLUMPY
    new_clumpy_angmom  = new_angmom_amsort[new_clumpy_mask]
    new_clumpy_am_dist = np.cumsum(new_clumpy_angmom)  
    if (len(new_clumpy_am_dist) >= 2):
        print(new_clumpy_am_dist)
        print(len(new_clumpy_am_dist))
        new_clumpy_am_dist_norm = new_clumpy_am_dist / new_clumpy_am_dist[len(n
ew_clumpy_am_dist)-1]
        plt.plot(new_clumpy_angmom,new_clumpy_am_dist_norm,'-',color='Green')
# SHOCKED    
    new_shock_angmom  = new_angmom_amsort[new_shock_mask]
    new_shock_am_dist = np.cumsum(new_shock_angmom)
    if (len(new_shock_am_dist) >= 2):
        print(new_shock_am_dist)
        print(len(new_shock_am_dist))
        new_shock_am_dist_norm = new_shock_am_dist / new_shock_am_dist[len(new_
shock_am_dist)-1]
        plt.plot(new_shock_angmom,new_shock_am_dist_norm,'-',color='Red')
# GALAXY GAS PLOTTING
    plt.title(str(len(new_iords_in_ts))+'new particles at z = '+str(redshift))
    plt.ylabel('Normalized Cumulative Distribution')
    plt.xlabel('Angular Momentum |j|')
    plt.xlim([0,25000])

####################################################################
# NEW GAS IN TIMESTEP GALAXY GAS --- BLACK HOLE GAS                #
####################################################################
# Reorder them in angmom order
    new_bh_amsort_ind    = np.argsort(new_bh_angmom_in_ts)
    new_bh_angmom_amsort = new_bh_angmom_in_ts[new_bh_amsort_ind]
    new_bh_iords_amsort  = new_bh_iords_in_ts[new_bh_amsort_ind]

# START PLOTTING
    new_bh_unshock_mask = np.in1d(new_bh_iords_amsort,h1_iords_unshock)
    new_bh_clumpy_mask  = np.in1d(new_bh_iords_amsort,h1_iords_clumpy)
    new_bh_shock_mask   = np.in1d(new_bh_iords_amsort,h1_iords_shock)
# UNSHOCKED
    new_bh_unshock_angmom  = new_bh_angmom_amsort[new_bh_unshock_mask]
    new_bh_unshock_am_dist = np.cumsum(new_bh_unshock_angmom)
    if (len(new_bh_unshock_am_dist) >= 2):
        print(new_bh_unshock_am_dist)
        print(len(new_bh_unshock_am_dist))
        new_bh_unshock_am_dist_norm = new_bh_unshock_am_dist / new_bh_unshock_a
m_dist[len(new_bh_unshock_am_dist)-1]
        plt.plot(new_bh_unshock_angmom,new_bh_unshock_am_dist_norm,'--',color='
Blue')
# CLUMPY
    new_bh_clumpy_angmom  = new_bh_angmom_amsort[new_bh_clumpy_mask]
    new_bh_clumpy_am_dist = np.cumsum(new_bh_clumpy_angmom)
    if (len(new_bh_clumpy_am_dist) >= 2):
        print(new_bh_clumpy_am_dist)
        print(len(new_bh_clumpy_am_dist))
        new_bh_clumpy_am_dist_norm = new_bh_clumpy_am_dist / new_bh_clumpy_am_d
ist[len(new_bh_clumpy_am_dist)-1]
        plt.plot(new_bh_clumpy_angmom,new_bh_clumpy_am_dist_norm,'--',color='Gr
een')
# SHOCKED
    new_bh_shock_angmom  = new_bh_angmom_amsort[new_bh_shock_mask]
    new_bh_shock_am_dist = np.cumsum(new_bh_shock_angmom)
    if (len(new_bh_shock_am_dist) >= 2):
        print(new_bh_shock_am_dist)
        print(len(new_bh_shock_am_dist))
        new_bh_shock_am_dist_norm = new_bh_shock_am_dist / new_bh_shock_am_dist
[len(new_bh_shock_am_dist)-1]
        plt.plot(new_bh_shock_angmom,new_bh_shock_am_dist_norm,'--',color='Red'
)

# BLACK HOLE GAS PLOTTING
    plt.savefig('angmom_cumudist_tsnew_'+ts_numbers[i]+'.ps')
    plt.clf()

    num_new_gas_in_ts.append(len(new_iords_in_ts))
    redshifts.append(redshift)

print('Final length of the angmom array',len(cumu_angmom),'should equal the ori
ginal number of not early iords',len(h1_iords_notearly))
print('Note: they are not in the same order')

# Save file for new gas number counts
datafile_path = "./ts_newgasnumbers.txt"
#datafile_id   = open(datafile_path, 'w+')
ts_numbers        = np.array(ts_numbers[1:])
ts_newgas_numbers = np.array(num_new_gas_in_ts)
redshifts         = np.array(redshifts)

print(len(ts_numbers))
print(len(redshifts))
print(len(ts_newgas_numbers))

#data = np.array([ts_numbers, redshifts, ts_newgas_numbers])
#data = data.T

#np.savetxt(datafile_id, data, fmt=['%d','%d','%d'])
#datafile_id.close()

np.savetxt(angmom_ts_numbers, ts_numbers)
np.savetxt(angmom_redshifts, redshifts)
np.savetxt(angmom_ts_newgas, ts_newgas_numbers)
\end{verbatim}

\subsection{infalltime.py}
\begin{verbatim}
# This script reads in all the clumpy particles that ever enter the SMBH and de
termines their 
# accretion times to plot a histogram that explicitly shows at what times the l
argest influxes
# of gas that end up in the SMBH entered the galaxy.

# Plan:
# Read in all gas particles accreted by h258:       grp1.allgas.iord.fits
# Read in corresponding accr times:                 grp1.accrz.fits
# Mask both above arrays using:                     clumpy.accr.iord.fits
# Create accrz array that corresponds to clumpy gas in main halo.
# Mask both clumpy array and clumpy accrz with:     ../centralbh/gasiords.good.
all
# Create a histogram of clumpy particles in SMBH.
# Plot against accretion times.
# (Also do this directly for clumpy gas in halo too, to compare.)
# Determine times of greatest gas influxes.

# Goal: Determine which mergers the gas particles from the biggest influxes com
e from.
# Trying to connect gas accertion of galaxy to gas accertion of 


# Vanderbilt Univ.  -- VPAC40:  /astro1/nicole/lowresh258/clumpygasaccr.py 
# N. Nicole Sanchez -- May 30, 2016 
from astropy.cosmology import w0waCDM as wCDM
import matplotlib.pyplot as plt
import pynbody as pyn
import pyfits  as pyf
import numpy   as np
import os.path

# Read in arrays
h1_iords_all    = pyf.getdata('../grp1.allgas.iord.fits')      # all gas IDs in
 h258 @ z=0
h1_gas_accrz    = pyf.getdata('../grp1.accrz.fits')            # accretion reds
hift  
h1_iords_clumpy = pyf.getdata('../clumpy.accr.iord.fits')      # all clumpy IDs
 in h258 @ z=0
h1_iords_smooth = pyf.getdata('../smooth.accr.iord.fits')
h1_iords_shocked = pyf.getdata('../shocked.iord.fits')
h1_iords_early  = pyf.getdata('../early.iord.fits') 
bh_iords_all    = np.loadtxt('../centralbh/gasiords.good.all',unpack=True) # al
l gas IDs in BH @ z=0
bh_iords_all    = bh_iords_all[0]

####### GAS PARTICLES HALO ENTRY TIMES #######
# Need to remove all early particles from allgas to compare to accrz
h1_early_mask     = np.in1d(h1_iords_all,h1_iords_early)
h1_notearly_mask  = ~h1_early_mask
h1_iords_notearly = h1_iords_all[h1_notearly_mask]

h1_gas_accrz_zsort = np.argsort(h1_gas_accrz)
h1_iords_notearly  = h1_iords_notearly[h1_gas_accrz_zsort]
h1_gas_accrz       = h1_gas_accrz[h1_gas_accrz_zsort]

# Mask arrays for clumpy in halo
h1_iords_clumpy_mask = np.in1d(h1_iords_notearly,h1_iords_clumpy)
h1_clumpy_iords      = h1_iords_notearly[h1_iords_clumpy_mask]
h1_clumpy_accrz      = h1_gas_accrz[h1_iords_clumpy_mask]

print('The number of clumpy from .fits: ',len(h1_iords_clumpy),' compared to th
ose found using a mask in the allgas.fits: ',len(h1_clumpy_iords),'These should
 be the same.')

# Make clumpy histogram for halo
#zbins = np.linspace(min(h1_clumpy_accrz),max(h1_clumpy_accrz),1000)

h1_clumpy_accrz_histogram = np.histogram(h1_clumpy_accrz,bins=40)
h1_clumpy_accrz_hist      = h1_clumpy_accrz_histogram[0]
h1_clumpy_accrz_bins      = h1_clumpy_accrz_histogram[1]
h1_clumpy_accrz_bins      = h1_clumpy_accrz_bins[:-1]

#plt.hist(h1_clumpy_accrz,bins=1000,drawstyle='steps')#,linestyle='Solid',color
='Green')
#plt.plot(h1_clumpy_accrz_bins,np.log10(h1_clumpy_accrz_hist),'-',color='Green'
)
#plt.xlim(max(h1_clumpy_accrz_bins),min(h1_clumpy_accrz_bins))
#plt.xlim(5,0)
#plt.title('Clumpy Gas Accretion Redshifts in h258 (Gasoline)')
#plt.xlabel('Redshift (bins=1000)')
#plt.ylabel('log(dN/dz)')
#plt.savefig('clumpy_accrzhist_gxy.ps')
#plt.show()
#plt.clf()
#quit()
# Mask arrays for clumpy in SMBH
bh_iords_clumpy_mask = np.in1d(h1_clumpy_iords,bh_iords_all)
bh_clumpy_iords      = h1_clumpy_iords[bh_iords_clumpy_mask]
bh_clumpy_accrz      = h1_clumpy_accrz[bh_iords_clumpy_mask]

print('The number of total bh accreted particles: ',len(bh_iords_all),' compare
d to the number of only clumpy particles in the bh: ',len(bh_clumpy_iords),' Th
is can be a fraction of 74%: ',float(len(bh_clumpy_iords))/len(bh_iords_all),',
 but might not because BH takes bites.')

# Make clumpy histogram for halo
#bhzbins = np.linspace(min(bh_clumpy_accrz),max(bh_clumpy_accrz),1000)

bh_clumpy_accrz_histogram = np.histogram(bh_clumpy_accrz,bins=100)
bh_clumpy_accrz_hist      = bh_clumpy_accrz_histogram[0]
bh_clumpy_accrz_bins      = bh_clumpy_accrz_histogram[1]
bh_clumpy_accrz_bins      = bh_clumpy_accrz_bins[:-1]

#plt.plot(bh_clumpy_accrz_bins,np.log10(bh_clumpy_accrz_hist),'--',color='Green
')
#plt.xlim(5,0)
#plt.title('Clumpy Gas Accretion Redshifts in the SMBH of h258 (Gasoline)')
#plt.xlabel('Redshift')
#plt.ylabel('log(dN/dz)')
#plt.savefig('clumpy_accrzhist_bh.ps')
#plt.show()
#plt.clf()

#plt.plot(h1_clumpy_accrz_bins,h1_clumpy_accrz_hist,'-',color='Green',label='Ha
lo Clumpy Gas Halo Accretion Redshift')
#plt.plot(bh_clumpy_accrz_bins,bh_clumpy_accrz_hist,'--',color='Green',label='S
MBH Clumpy Gas Halo Accretion Redshift')
#plt.xlim(max(h1_clumpy_accrz_bins),min(h1_clumpy_accrz_bins))
#plt.xlim(5,0)
#plt.ylim(0,1000000)
#plt.yscale('log')
#plt.title('Clumpy Gas Accretion Redshifts in SMBH and Halo of h258 (Gasoline)'
)
#plt.xlabel('Redshift')
#plt.ylabel('dN/dz')
#plt.legend(loc=2)
#plt.savefig('clumpy_accrzhist_bh_gxy.ps')
#plt.show()
#plt.clf()
#quit()


##### START TIME CONVERSION #####

# Convert redshift to Gyr
H_0     = 73 
Omega_m = 0.24
Omega_L = 0.76
present = 13.721 # Gyr since Big bang
cosmodel = wCDM(H_0, Omega_m, Omega_L)
bh_lb_time = cosmodel.lookback_time(bh_clumpy_accrz)
bh_clumpy_accrt  = present - bh_lb_time


#if os.path.isfile('h1_clumpy_accrt.fits') == True :
#    h1_clumpy_accrt = pyf.open('h1_clumpy_accrt.fits')
#    print 'read in h1 clumpy accrt file'
#else:
h1_lb_time = cosmodel.lookback_time(h1_clumpy_accrz) 
h1_clumpy_accrt  = present - h1_lb_time
#    hdu = pyf.PrimaryHDU(h1_clumpy_accrt,header=None)
#    hdu.writeto('h1_clumpy_accrt.fits')
#print h1_clumpy_accrt[0:100]

#quit()

##### TIME [Gyr] #####
# Make clumpy histograms for halo entry times (for both gxy and halo) 
#tbins = np.linspace(min(h1_clumpy_accrt),max(h1_clumpy_accrt),10)
print 'Do these ',h1_clumpy_iords[0:10],' match these? ',bh_clumpy_iords[0:10],
'? They do not need to, because h1 (the first array) is considering ALL the clu
mpy particles entering the halo, and bh (the second array) is only considering 
the subset of the first array that eventually ends up in the SMBH.'
#quit()

# Turns out the above arrays are NOT in the same order SO:
#h1_clumpy_iordsort = np.argsort(h1_clumpy_iords)
#h1_clumpy_iords    = h1_clumpy_iord[h1_clumpy_iordsort]
#h1_clumpy_accrt    = h1_clumpy_accrt[h1_clumpy_iordsort]

#bh_clumpy_iordsort = np.argsort(bh_clumpy_iords)
#bh_clumpy_iords    = bh_clumpy_iords[bh_clumpy_iordsort]
#bh_clumpy_accrt    = bh_clumpy_accrt[bh_clumpy_iordsort]

#print 'They do not match above so we reorder them and check again: So do these
 ',h1_clumpy_iords[0:10],' match these? ',bh_clumpy_iords[0:10]
#quit()

# Now we make histograms of the accrt and eat times
h1_clumpy_accrt_histogram = np.histogram(h1_clumpy_accrt,bins=30)
h1_clumpy_accrt_hist      = h1_clumpy_accrt_histogram[0]
h1_clumpy_accrt_bins      = h1_clumpy_accrt_histogram[1]
h1_clumpy_accrt_bins      = h1_clumpy_accrt_bins[:-1]

bh_clumpy_accrt_histogram = np.histogram(bh_clumpy_accrt,bins=25)
bh_clumpy_accrt_hist      = bh_clumpy_accrt_histogram[0]
bh_clumpy_accrt_bins      = bh_clumpy_accrt_histogram[1]
bh_clumpy_accrt_bins      = bh_clumpy_accrt_bins[:-1]

# Note to self:  This plot is working for Gas h258 and lovely, don't touch
plt.plot(h1_clumpy_accrt_bins,np.log10(h1_clumpy_accrt_hist),'-',color='Green',
label='Halo Clumpy Gas Halo Accr. Time ('+np.str(len(h1_clumpy_accrt))+')')
plt.plot(bh_clumpy_accrt_bins,np.log10(bh_clumpy_accrt_hist),'--',color='Green'
,label='SMBH Clumpy Gas Halo Accr. Time ('+np.str(len(bh_clumpy_accrt))+')')
#plt.xlim(max(bh_clumpy_accrt_bins),min(bh_clumpy_accrt_bins))
#plt.xlim(2,13.5)
plt.ylim(0,8)
#plt.yscale('log')
plt.title('Clumpy Gas Accretion Time of h258 (Gasoline)')
plt.xlabel('Halo Accretion Time (Gyr)')
plt.ylabel('log(dN/dz)')
plt.legend()
plt.savefig('clumpy_accrthist_bh_gxy_bin25_30.ps')
#plt.show()
plt.clf()
#quit()


####### GAS PARTICLES BH ENTRY TIMES #######
# For high res:
#bh_iords_allmulti = pyf.getdata('../centralbh/centralbh_gasiords.fits')
#bh_iords_eatenz   = pyf.getdata('../centralbh/centralbh_redshift.fits')

bh_iords_allmulti_allfile = np.loadtxt('../centralbh/out.distance',usecols=(3,5
))
bh_iords_allmulti_allfile = np.transpose(bh_iords_allmulti_allfile)

bh_iords_allmulti = bh_iords_allmulti_allfile[1]
bh_iords_eatenst  = bh_iords_allmulti_allfile[0]
# Not redshift! Simulation time.

tunit = (1.223128407*10**18)/(3.1556926*10**7)/(10**9) # Gyr
bh_iords_eatent   = np.multiply(bh_iords_eatenst,tunit)   

# To get the redshifts, you'll need to read in out.mdot and then mask eattime t
o time and
# then compare that to a and solve for z  (;z = 1./a-1. ; redshift.) ignoring f
or now because
# time

# I want to get the time of the first accretion of the gas particle into the bl
ack hole. 
# bh_iords:          tells me the number of unique gas iords that enter the SMB
H
# bh_iords_allmulti: tells me the number of gas iords that enter and keep getti
ng bitten
#                    by the black hole (so there are multiples of the gas iords
) 
# bh_iords_eatenz:   comparable to allmulti; but we want the time when the firs
t bite is 
#                    taken out of each particle so we need a loop to pull out t
hat time

# EXAMPLE:
# >>> t = np.array([2,3,4,6,5,7,8,9,1,10])
# >>> x = np.array([1,2,1,2,3,3,3,4,5,3,])
# >>> sort_t_ind = np.argsort(t)
# >>> x_sort = x[sort_t_ind]
# [5 1 2 1 3 2 3 3 4 3]
# >>> t_sort = t[sort_t_ind]
# [1 2 3 4 5 6 7 8 9 10]
# >>> sort_x_ind = np.argsort(x_sort)
# >>> x_sort_x = x_sort[sort_x_ind]
# [1 1 2 2 3 3 3 3 4 5]
# >>> t_sort_x = t_sort[sort_x_ind]
# [2 4 3 6 5 7 8 10 9 1]
# >>> only_first_x_appear = [0]
# >>> for i in range(1,len(x_sort_x)):
# ...     if x_sort_x[i] != x_sort_x[i-1]:
# ...             only_first_x_appear.append(i)
# ... 
# >>> print only_first_x_appear
# [0, 2, 4, 8, 9]
# >>> print x_sort_x[only_first_x_appear]
# [1 2 3 4 5]
# >>> print t_sort_x[only_first_x_appear]
# [2 3 5 9 1]
# These last two lines are what we want because we can compare this to bh_iords
 and 
# get the first eaten time for each gas particle.
# NOTE: These are already in chronological order! TRUE FOR CHANGA
# print bh_iords_eatenz[0:100]

# Not sure all the low res particles are in time order
bh_iords_allmulti_tsort_ind = np.argsort(bh_iords_eatent)
bh_iords_allmulti_tsort = bh_iords_allmulti[bh_iords_allmulti_tsort_ind]
bh_iords_eatent_tsort   = bh_iords_eatent[bh_iords_allmulti_tsort_ind]

bh_iords_allmulti_iordsort_ind = np.argsort(bh_iords_allmulti)
bh_iords_allmulti_iordsort = bh_iords_allmulti[bh_iords_allmulti_iordsort_ind]
bh_iords_eatent_iordsort   = bh_iords_eatent[bh_iords_allmulti_iordsort_ind]

bh_iords_allmulti_uniq_ind = [0]
for i in range(1,len(bh_iords_allmulti_iordsort)):
    if bh_iords_allmulti_iordsort[i] != bh_iords_allmulti_iordsort[i-1]:
        bh_iords_allmulti_uniq_ind.append(i)

bh_iords_allmulti_uniq = bh_iords_allmulti_iordsort[bh_iords_allmulti_uniq_ind]
bh_iords_eatent_uniq   = bh_iords_eatent_iordsort[bh_iords_allmulti_uniq_ind]

print('So the number of unique iords in the iords_allmulti is',len(bh_iords_all
multi_uniq))
print('While the number of unique iords in the iords from gasiords.good is', le
n(bh_iords_all))
print('We need these iords_allmulti ',bh_iords_allmulti_uniq[0:10],' to match t
hese gasiords.good iords ',bh_iords_all[0:10],' to confirm that we are taking t
he differences of the right particles later.')
# The above arrays just confirm the number of total BH accreted particles read 
in by two different sources. gasiords.good however does not have comparable acc
retion times; but iords_allmulti do. But we read in iords_allmulti in a way tha
t included ALL the bites that the BH took out of each particle. We compare the 
final length of gasiords.good and iords_allmulti to make sure I have the right 
number of unique particles (and including the initial bite times).
# THESE ARE THE SAME WHOO. Thank goodness. Ok, so we don't have to mask.
#quit()

# Necessary for high res h258 which has rando particles; but not necessary here
# Now we need to mask allmulti uniq using bh_iords to get the eatentimes for th
at gas
#bh_halo_times_mask  = np.in1d(h1_iords_notearly,bh_iords_all)
#bh_halo_iords       = h1_iords_notearly[bh_halo_times_mask]
#bh_haloentry_accrz  = h1_gas_accrz[bh_halo_times_mask]
#bhentry_iords_mask  = np.in1d(bh_iords_all,h1_iords_notearly)
#bhentry_iords       = bh_iords_allmulti_uniq[bhentry_iords_mask]
#bhentry_eatt       = bh_iords_eatent_uniq[bhentry_iords_mask]
#print('To match the halo iords',len(bh_halo_iords),' we must  remove those 6 w
eird particles (one of which is early) and make sure the # of bh gas iords matc
hes: ',len(bhentry_iords))
#print('We masked our iords to their accretion/eat times so let us make sure th
e iods still match: Do these iords ',bh_halo_iords[0:10],' match these? ',bhent
ry_iords[0:10])
#quit()

# Mask arrays for clumpy in BH entry; Already done for the halo ones: bh_clumpy
_iords
bh_iords_clumpy_mask = np.in1d(bh_iords_allmulti_uniq,h1_iords_clumpy)
bh_clumpy_eatiords   = bh_iords_allmulti_uniq[bh_iords_clumpy_mask]
bh_clumpy_eatt       = bh_iords_eatent_uniq[bh_iords_clumpy_mask]

print('So THESE really are the array that have to match iords: bh_clumpy_iords:
 ',bh_clumpy_iords[0:20],' and bh_clumpy_eatiords: ',bh_clumpy_eatiords[0:20],'
 because the comparable time arrays are what we will subtract to get dt.')
print('These arrays must also be the same lengths: Does ',len(bh_clumpy_iords),
' equal ',len(bh_clumpy_eatiords))

# And we know from getting the unique iords for bh_clumpy_eatiords that they ar
e in iord order
# so we're going to reorder bh_clumpy_iord (the clumpy iords comparing to the h
alo accrt) into 
# iord order as well
bh_clumpy_iordsort = np.argsort(bh_clumpy_iords)
bh_clumpy_haloiord = bh_clumpy_iords[bh_clumpy_iordsort]
bh_clumpy_accrt    = bh_clumpy_accrt[bh_clumpy_iordsort]

print('Check that these are now matched: halo accr iords ',bh_clumpy_haloiord[0
:10],' and bh eaten iords ',bh_clumpy_eatiords[0:10])
#quit()

##### TIME [Gyr] #####
# Make clumpy histograms for bh entry times (for both gxy and halo)

# This is a repeat of the same one from above; necessary because of reorder of 
times 
bh_clumpy_accrt_histogram = np.histogram(bh_clumpy_accrt,bins=25)
bh_clumpy_accrt_hist      = bh_clumpy_accrt_histogram[0]
bh_clumpy_accrt_bins      = bh_clumpy_accrt_histogram[1]
bh_clumpy_accrt_bins      = bh_clumpy_accrt_bins[:-1]

bh_clumpy_eatt_histogram = np.histogram(bh_clumpy_eatt,bins=100)#,bins=bheattbi
ns)
bh_clumpy_eatt_hist      = bh_clumpy_eatt_histogram[0]
bh_clumpy_eatt_bins      = bh_clumpy_eatt_histogram[1]
bh_clumpy_eatt_bins      = bh_clumpy_eatt_bins[:-1]

plt.plot(bh_clumpy_accrt_bins,np.log10(bh_clumpy_accrt_hist),'-',color='Green',
label='Clumpy Gas Accr. by Halo ('+np.str(len(bh_clumpy_accrt))+' particles)')
plt.plot(bh_clumpy_eatt_bins,np.log10(bh_clumpy_eatt_hist),'--',color='Green',l
abel='Clumpy Gas Eaten by BH ('+np.str(len(bh_clumpy_eatt))+' particles)')
plt.ylim(0,6)
#plt.xlim(2,13.5)
plt.title('Clumpy Gas in SMBH of h258 (Gasoline)')
plt.xlabel('Accretion Time (Gyr)')
plt.ylabel('log(dN/dt)')
plt.legend()
plt.savefig('clumpy_bheatt_and_accrthist_bh_gxy_bin100_25.ps')
plt.show()
plt.clf()
#quit()

###### Gas Infall Time (Halo - BH entry times) ######
dt_accrt    = np.subtract(bh_clumpy_eatt,bh_clumpy_accrt)
negative_dt = dt_accrt[np.where(dt_accrt < 0)]
#print bh_clumpy_accrt[0:10]
#print bh_clumpy_eatt[0:10]
print 'Difference between eatt and accrt',dt_accrt[0:100]
print 'There should be no negative values: Min dt_accrt',min(dt_accrt),' and ho
w many negative values ',len(negative_dt)
print negative_dt
print bh_clumpy_haloiord[np.where(dt_accrt < 0)][0:10]
print bh_clumpy_accrt[np.where(dt_accrt < 0)][0:10]
print bh_clumpy_eatt[np.where(dt_accrt < 0)][0:10]
print bh_clumpy_eatiords[np.where(dt_accrt < 0)][0:10]

dt_accrt_histogram = np.histogram(dt_accrt,bins=100)
dt_accrt_hist      = dt_accrt_histogram[0]
dt_accrt_bins      = dt_accrt_histogram[1]
dt_accrt_bins      = dt_accrt_bins[:-1]

plt.plot(dt_accrt_bins,np.log10(dt_accrt_hist),'-',color='Green')
plt.ylim(0,3)
plt.xlim(0,13.5)
plt.title('Clumpy Gas Infall Time of h258 (Gasoline)')
plt.xlabel('Infall Time (Gyr)')
plt.ylabel('Log(dN/dt)')
plt.savefig('clumpy_infalltime_bins100.ps')
plt.show()
plt.clf()

plt.plot(dt_accrt,bh_clumpy_accrt,'.',color='Green')
plt.title('Clumpy Gas Infall Time vs Halo Accretion Time of h258 (Gasoline)')
plt.xlabel('Halo Accretion Time (Gyr)')
plt.ylabel('Infall Time (Gyr)')
plt.ylim(0,13.5)
plt.xlim(0,13.5)
plt.savefig('clumpy_infalltimevsaccrt_bins100.ps')
plt.show()
plt.clf()
\end{verbatim}


\thispagestyle{empty}

\chapter{APPENDIX}
\thispagestyle{empty}
%\addtocontents{toc}{\protect\addvspace{10pt}}
%\addtocontents{toc}{\textbf{APPENDIX}}

\chapter*{ACKNOWLEDGEMENTS}
\thispagestyle{empty}
\addtocontents{toc}{\protect\addvspace{10pt}}
\addtocontents{toc}{\textbf{ACKNOWLEDGEMENTS}}

\renewcommand{\thepage}{}\renewcommand{\thepage}{}
\bibliography{/Users/the_neekster/Documents/MENDELEY/MEND_bibtexfiles/Sanchez2016.bib}


\end{document}
